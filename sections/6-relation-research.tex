\section{関連研究}
関連研究として,ユーザの移動に注目したものでmiyata氏のものがある.
これはユーザの移動によってWi-Fiの通信品質を改善しようとするものであるが,これはユーザを最大限移動させて通信環境を良化させるものであった.
また,チャネル使用率について,推定しようとする研究はある.
福原ら[X]は、プローブパケットを使用せずに無線LANのチャネル占有率を推定する受動的手法を提案した。
彼らの手法は、無線LAN無線リソースの利用率指標として時間占有率を定義し、実際の送信時間のみを考慮する従来の測定手法の限界に対処する。
提案手法Method3は、インターフレーム間隔(IFS)とバックオフ期間を含めて総占有時間を算出し、データパケットとビーコン内のシーケンス番号を活用することでパケット損失を補正する。
3つの測定手法を比較した実験により、解析装置と通信リンク間の距離が増加すると従来手法の性能が低下する一方、Method3は著しいパケット損失条件下でもチャネル占有率を$98\%$以上の精度で推定できることを実証した。