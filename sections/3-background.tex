%%%%%%%%%% 背景知識 %%%%%%%%%%
\section{背景知識}
\subsection{無線LANの基礎技術}
無線LAN(Wireless Local Area Network)は, IEEE 802.11標準規格に基づく無線通信技術により構築されるネットワークである. 
システムは主にアクセスポイント(AP: Access Point)とクライアント端末から構成される.
APが2.4GHz帯または5GHz帯(IEEE 802.11ax以降では6GHz帯も利用可能)の電波を用いてクライアントとの双方向通信を実現する.
近年の規格発展により, IEEE 802.11ac(Wi-Fi 5)では最大6.9Gbps, IEEE 802.11ax(Wi-Fi 6)では最大9.6Gbpsの理論スループットが達成されている. 
しかし, 実環境では後述する様々な要因により, 実効スループットは理論値を大きく下回ることが知られている.
\subsection{無線LANの通信方式}
無線LANは, CSMA/CA(Carrier Sense Multiple Access with Collision Avoidance)方式を採用している. 
この方式では, 端末が通信を開始する前にチャネルの状態を監視し, 他の端末が通信中でないことを確認してから送信を行う. 
もしチャネルが使用中であれば, ランダムなバックオフ時間を待機した後に再度チャネルの状態を確認する. 
これにより, 複数の端末が同時に通信を試みた際の衝突を回避し, 効率的なチャネル利用を実現している.
さらに, ACK(acknowledgment)フレームを用いた確認応答機能により, 送信データの正確な受信を保証している. 
受信側は正常にデータを受信した場合にACKフレームを送信し, 送信側はこれを受信することでデータの到達を確認する. 
ACKが受信されない場合, 送信側は再送を行うことで信頼性の高い通信を実現している.
\subsection{無線LANのチャネルと周波数帯域}
無線LANは, 2.4GHz帯と5GHz帯の周波数帯域を利用して通信を行う. 
2.4GHz帯は, 電波の到達距離が長く, 障害物に対する透過性が高い一方で, 他の無線機器(電子レンジ,Bluetooth機器など)との干渉が発生しやすい. 
一方, 5GHz帯は, より広い帯域幅を提供し, 高速な通信が可能であるが, 電波の到達距離が短く, 障害物による減衰が大きいという特性がある.
無線LANでは, 複数のチャネルが定義されており, 各チャネルは特定の周波数範囲を占有する.
2.4GHz帯では, 通常1~14チャネルが利用可能であり, 各チャネルは20MHzの帯域幅を持つ. 
5GHz帯では, より多くのチャネルが利用可能であり, 20MHz, 40MHz, 80MHz, 160MHzの帯域幅を持つチャネルが存在する. 
チャネル選択は, 干渉の回避と通信品質の最適化において重要な役割を果たす.

\subsection{無線LANにおける通信品質のパラメータ}
無線LAN環境における通信品質は, 以下の主要なパラメータが存在する.
\begin{enumerate}
    \item スループット
    単位時間あたりに転送可能なデータ量(bps)であり, この値が小さくなるとユーザ体感品質が低下する.
    \item 遅延(レイテンシ)
    データ送信から受信完了までの時間(ms)を表し, リアルタイム通信において特に重要となる.
    この値が大きくなると,ライブ配信やWeb会議等のリアルタイム性を要する通信に支障が生じる.
    \item パケットロス率
    送信されたパケットのうち, 正常に受信されなかった割合(\%)を示す. 再送制御の増加により遅延やスループット低下の原因となる.
    \item 受信信号強度(RSSI)
    受信電波の強度(dBm)を表し, 通信可能範囲や接続安定性の指標となる.
    一般に-70dBm以上が良好, -80dBm以下で通信品質が著しく低下する.
    \item 信号対雑音比(SNR)
    信号強度と雑音レベルの比(dB)であり, 高いSNRはより高速な変調方式の使用を可能にする.
    \item チャネル使用率
    チャネルの使用状況を示すパラメータであり, 通信品質に影響を与える. 通常はパーセント表示される.
\end{enumerate}

\subsection{【技術・概念2】}
【技術や概念に関する説明】

\subsection{【技術・概念3】}
【技術や概念に関する説明】
