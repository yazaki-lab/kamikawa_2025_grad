%%%%%%%%%% 背景知識 %%%%%%%%%%
\section{背景知識}
\subsection{無線LANの基礎技術}
無線LAN(Wireless Local Area Network)は, IEEE 802.11標準規格に基づく無線通信技術により構築されるネットワークである. 
システムは主にアクセスポイント(AP: Access Point)とクライアント端末から構成される.
APが2.4GHz帯または5GHz帯(IEEE 802.11ax以降では6GHz帯も利用可能)の電波を用いてクライアントとの双方向通信を実現する.
近年の規格発展により, IEEE 802.11ac(Wi-Fi 5)では最大6.9Gbps, IEEE 802.11ax(Wi-Fi 6)では最大9.6Gbpsの理論スループットが達成されている. 
しかし, 実環境では後述する様々な要因により, 実効スループットは理論値を大きく下回ることが知られている.

\subsection{【技術・概念2】}
【技術や概念に関する説明】

\subsection{【技術・概念3】}
【技術や概念に関する説明】
