%%%%%%%%%% 背景知識 %%%%%%%%%%
\section{背景知識}
\subsection{無線LANの基礎技術\cite{nttwest2024},\cite{Wifistandard}}
無線LAN(Wireless Local Area Network)は, IEEE 802.11標準規格に基づく無線通信技術により構築されるネットワークである.
システムは主にアクセスポイント(AP: Access Point)とクライアント端末から構成される.
APが2.4GHz帯または5GHz帯(IEEE 802.11ax以降では6GHz帯も利用可能)の電波を用いてクライアントとの双方向通信を実現する.
近年の規格発展により, IEEE 802.11ac(Wi-Fi 5)では最大6.9Gbps, IEEE 802.11ax(Wi-Fi 6)では最大9.6Gbpsの理論スループットが達成されている.
しかし, 実環境では後述する様々な要因により, 実効スループットは理論値を大きく下回ることが知られている.
\subsection{無線LANの通信方式\cite{Wifistandard}}
無線LANは, Carrier Sense Multiple Access with Collision Avoidance(CSMA/CA)方式を採用している.
この方式では, 端末が通信を開始する前にチャネルの状態を監視し, 他の端末が通信中でないことを確認してから送信を行う.
もしチャネルが使用中であれば, ランダムなバックオフ時間を待機した後に再度チャネルの状態を確認する.
これにより, 複数の端末が同時に通信を試みた際の衝突を回避し, 効率的なチャネル利用を実現している.
さらに, Acknowledgment(ACK)フレームを用いた確認応答機能により, 送信データの正確な受信を保証している.
受信側は正常にデータを受信した場合にACKフレームを送信し, 送信側はこれを受信することでデータの到達を確認する.
ACKが受信されない場合, 送信側は再送を行うことで信頼性の高い通信を実現している.
\subsection{無線LANのチャネルと周波数帯域\cite{Wifistandard}}
無線LANは, 2.4GHz帯と5GHz帯,6GHz帯の周波数帯域を利用して通信を行う.
2.4GHz帯は, 電波の到達距離が長く, 障害物に対する透過性が高い一方で, 他の無線機器(電子レンジ,Bluetooth機器など)との干渉が発生しやすい\cite{FURUNO2024}.%参考文献
一方, 5GHz帯は, より広い帯域幅を提供し, 高速な通信が可能であるが, 電波の到達距離が短く, 障害物による減衰が大きいという特性がある.
無線LANでは, 複数のチャネルが定義されており, 各チャネルは特定の周波数範囲を占有する.
2.4GHz帯では, 通常1~14チャネルが利用可能であり, 各チャネルは20MHzの帯域幅を持つ.
5GHz帯では, より多くのチャネルが利用可能であり, 20MHz, 40MHz, 80MHz, 160MHzの帯域幅を持つチャネルが存在する.
チャネル選択は, 干渉の回避と通信品質の最適化において重要な役割を果たす.

\subsection{無線LANにおける通信品質のパラメータ\cite{Wifistandard}}
無線LAN環境における通信品質は, 以下の主要なパラメータが存在する.
\begin{itemize}
    \item スループット (Throughput) \\
    単位時間あたりに実際に転送可能なデータ量(bps)を指す.
    理論上の最大通信速度(帯域幅)とは異なり,プロトコルのオーバーヘッドやネットワークの混雑状況を加味した実効速度である.
    この値の低下は,ファイル転送時間の増大や動画の低画質化など,ユーザ体感品質(QoE:Quality of Experience)の低下に直結する.

    \item 遅延(レイテンシ / Latency) \\
    パケットが送信元から宛先へ到達するまでに要する時間(ms).
    双方向通信においては往復時間(RTT: Round Trip Time)で評価される.
    リアルタイム通信において極めて重要であり,この値が増大すると,Web会議の音声途切れやオンラインゲームの操作遅延など,即時性を要するアプリケーションの品質を損なう.

    \item パケット損失率 (Packet Loss Rate)\\
    送信されたパケットが,輻輳(混雑)やノイズ,伝送路の不安定さによって正常に受信されない割合.
    損失が発生すると,TCP等のプロトコルによる再送制御が作動し,結果として遅延の増大や実効スループットの低下を招く.

    \item 受信信号強度 (RSSI: Received Signal Strength Indicator) \\
    受信した電波の強度を表す指標(dBm).値が0に近いほど信号が強く,通信可能範囲や接続の安定性を判断する物理的な指標となる.
    ただし,強度が十分であっても後述するSNRが低い場合は,安定した通信が困難になることがある.

    \item 信号対雑音比 (SNR: Signal-to-Noise Ratio) \\
    所望の信号電力と背景雑音電力の比(dB).
    SNRが高いほど信号がノイズに埋もれずクリアであることを示し,結果としてスループットが向上する.

    \item チャネル使用率 (Channel Utilization) \\
    特定の周波数チャネルが単位時間のうちに占有されている時間の割合(\%).
    無線LAN等の共有媒体においては,他端末や干渉波による使用率が高いほどスループットの低下か遅延の原因となる.
\end{itemize}

\subsection{チャネル使用率}
チャネル使用率は, 無線LANチャネルが占有されている割合を示す指標であり, 通信品質に大きな影響を与える.
チャネル使用率は, 他の端末による通信が頻繁に発生している時や,外乱・電波干渉が起こっている時などに高くなり,新たな通信の開始が困難になる.
これにより, スループットの低下, 遅延の増加, パケットロス率の上昇といった通信品質の劣化が生じる.
チャネル使用率は, 通常0\%から100\%の範囲で表され, 100\%に近い値はチャネルがほぼ常に占有されている状態を示す.\\
チャネル使用率の測定は, 無線LAN機器や専用の解析ツールを用いて行われる.
これらのツールは, チャネルの占有時間を監視し, 使用率をリアルタイムで算出する機能を備えている.
ネットワーク管理者は, チャネル使用率の情報を活用して, チャネルの最適化や負荷分散の施策を講じることが可能である.
チャネル使用率$U_{ch}$は,チャネル使用状況を測定できた時間のうち,実際にチャネルが使用されていた時間の割合として定義される.


\subsection{AP密集環境における干渉問題}
エンタープライズ無線LAN環境では,通信容量の確保やカバレッジの向上を目的として,同一環境内に多数のAPが配置される.
しかし,APが高密度に配置される環境では,隣接チャネル干渉や同一チャネル干渉が発生しやすくなり,通信品質の低下を招く要因となる.

同一チャネル干渉は,複数のAPが同一のチャネルを使用することにより発生し,CSMA/CA方式に基づく送信待ち時間の増加を引き起こす.
一方,隣接チャネル干渉は,周波数帯域が部分的に重複するチャネルを使用する場合に発生し,パケット誤り率の増加や再送の増大につながる.
これらの干渉は,物理的な距離が十分に離れていないAP間で顕著となり,結果としてスループットや遅延特性に悪影響を及ぼす.


\subsection{管理外APによる影響}
近年の無線LAN環境では,管理者が設置・管理していない通信機器が通信品質に影響を与えるケースが増加している.
具体的には,来訪者が持ち込むモバイルルータやスマートフォンのテザリング機能などが挙げられる.
これらの管理外APは,既存の無線LANシステムと同一または隣接するチャネルを使用する場合が多く,予期せぬチャネル使用率の増加を引き起こす.

管理外APはネットワーク管理者からは直接制御することができず,その存在や通信状況を正確に把握することが困難である.
そのため,APと端末間の距離や受信信号強度のみを基準とした接続制御では,実際の通信品質を適切に評価できない場合がある.

\subsection{AP選択方式とその課題}
無線LANにおけるAP選択は,クライアント端末がどのAPに接続するかを決定する重要なプロセスである.
一般的な端末では,受信信号強度(RSSI)が最も高いAPを選択する方式が広く用いられている.
この方式は実装が容易である一方で,APの混雑状況やチャネル使用率を考慮していないという課題がある.

その結果,多数の端末が特定のAPに集中し,通信品質の低下や負荷の偏りが発生することが知られている.
このような問題は,APの設置間隔が短いエンタープライズ環境において特に顕著であり,RSSIのみを用いたAP選択の限界を示している.

\subsection{ユーザ移動と通信品質の関係}
無線LAN環境における通信品質は,ユーザ端末の位置に大きく依存する.
ユーザがAPに近づくことで受信信号強度やSNRが向上し,高速な変調方式の利用が可能となる.
一方で,AP周辺は多くのユーザが集中しやすく,チャネル使用率の増加によって必ずしも高いスループットが得られるとは限らない.

また,ユーザのわずかな移動によっても,接続APや利用チャネルが変化し,通信品質が大きく変動する場合がある.
このことから,ユーザの位置情報と無線環境情報を組み合わせて評価することが,通信品質の最適化において重要であるといえる.

\subsection{無線LANにおける負荷分散の考え方}
無線LAN環境では,特定のAPやチャネルに通信負荷が集中することを避けるため,負荷分散が重要な課題となる.
従来の負荷分散手法は,主にAP側での送信制御や接続制限といった管理者主導の手法が中心であった.
しかし,これらの手法はリアルタイム性や利用者体感品質の観点から限界がある.

近年では,利用者の行動や接続選択がネットワーク全体の性能に影響を与えるという観点から,ユーザを含めた負荷分散の重要性が指摘されている.
このような背景から,ネットワーク側の情報を活用してユーザに適切な行動を促す手法が注目されている.

\subsection{シミュレーションツール ns-3\cite{ns-3}}

ns-3 は離散事象ネットワークシミュレータであり,インターネットシステム全般の挙動を模擬するためのオープンソースソフトウェアである.
本ツールは主に研究および教育目的で設計されており,GNU GPLv2 ライセンスの下で公開されている.
ns-3 の基本的な考え方は,実際のネットワークで発生する出来事(イベント)を順序立てて処理し,応答や遅延,通信プロトコルのふるまいを詳細に再現する点にある.

具体的には,ns-3 はネットワークノード,プロトコルスタック,通信チャネルといった要素をソフトウェア内部に構築し,イベントスケジューラを用いて各イベントを処理する.
また,C++ を主言語として使用し,Python バインディングを通じてスクリプト形式でも操作可能な柔軟性を持つ.

Wi-Fi のシミュレーションに関しては,ns-3 が IEEE 802.11 準拠のモデルを標準で提供しているため,
アクセスポイントと端末間の通信や,基本的な物理層・MAC 層の挙動を模擬できる.
具体的には,インフラストラクチャーモードおよびアドホックモードの双方がサポートされ,802.11a/b/g/n などの物理層仕様,伝搬損失モデル,EDCA による QoS 拡張などを設定可能である.

このような特性から,Wi-Fi ネットワークの性能評価やプロトコル設計,トラフィック挙動の解析といった研究課題において,ns-3 は非常に有用なツールである.
実ネットワークを構築せずに多様なシナリオを再現できる点は,コストや時間の節約にもつながる.