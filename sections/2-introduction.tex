\section{はじめに}

\subsection{研究の背景}
現代社会において,インターネットは社会インフラとして不可欠な存在となっている.
スマートフォンやタブレット,ノートパソコンなどのモバイル端末の普及に伴い,人々がインターネットに接続する機会は飛躍的に増加し,もはや日常生活のあらゆる場面でネットワーク接続が求められるようになった.
特に,カフェやレストラン,図書館,商業施設,イベント会場,空港などの公共空間においては,無線LAN環境の提供が利用者にとっての重要な選択基準の一つとなっており,Wi-Fi設備は事実上の標準装備として認識されるに至っている.\\
このような状況を受け,多くの施設運営者はWi-Fi環境の整備に積極的に取り組んできた.
アクセスポイントの増設,高性能な通信機器の導入,広帯域な回線の契約など,設備面での投資は年々拡大している.しかし,これらの設備投資にもかかわらず,実際の利用現場では「Wi-Fiが繋がりにくい」「通信速度が遅い」「接続が頻繁に切れる」といった利用者からのクレームや不満が後を絶たない.
特に,多数の利用者が同時にネットワークを利用する状況(大規模なイベント会場,セミナー会場,人気カフェの混雑時など)においては,通信品質の著しい低下が頻繁に報告されている.\\
現在,こうした問題に対処するため,ネットワークオペレーションセンター(NOC)をはじめとする管理者側は様々な技術的施策を講じている.
帯域制御による公平な帯域分配,QoS(Quality of Service)設定による優先制御,チャンネル割り当ての最適化,干渉源の特定と除去,アクセスポイントの動的な負荷分散など,様々なネットワーク管理技術が使われている.
また,リアルタイムでのトラフィック監視やログ解析,異常検知システムの導入など,運用管理の自動化・効率化も進められてきた.
しかしながら,これらの管理者側の努力にもかかわらず,通信品質の問題は根本的な解決には至っていないのが現状である.\\
その理由の一つとして,従来のアプローチが主にネットワーク機器やインフラ側の最適化に焦点を当てており,実際にネットワークを利用する「ユーザ」の行動や振る舞いについては十分に考慮されてこなかったことが挙げられる.
例えば,利用者が一箇所に集中してしまうことによるアクセスポイントへの負荷の偏り,不要なアプリケーションのバックグラウンド通信による帯域の圧迫,同時に大容量ファイルのダウンロードを開始することによる輻輳の発生など,利用者の行動パターンがネットワーク全体のパフォーマンスに大きな影響を与えている事例は数多く存在する.
通信環境の問題を技術的側面からのみ捉えるのではなく,通信システムを構成する一要素として「ユーザ」を位置づけ,利用者の行動も含めた全体最適を図る必要があると考えられる.

\subsection{研究の目的}
本研究の目的は,従来のネットワーク管理の枠組みを拡張し,人間を通信環境を構成する重要な要素として明示的に捉えることで,通信品質の改善を実現することにある.
具体的には,利用者に対して適切な行動を促すことにより,ネットワーク全体のパフォーマンスを向上させる手法を提案し,その有効性を検証する.\\
従来の通信品質改善アプローチは,ネットワーク機器の性能向上や設定の最適化といった技術的施策に依存してきた.
しかし,いかに高性能な機器を導入し,精緻な制御アルゴリズムを実装したとしても,利用者の非効率的な行動や無自覚な振る舞いがボトルネックとなっている場合には,期待される効果は得られない.
本研究では,この問題意識に基づき,人間を通信システムの受動的な利用者としてではなく,能動的に通信品質に影響を与える主体として位置づける.
具体的には,利用者の行動を適切に誘導することで,以下のような通信品質の改善効果が期待できる.\\
第一に,空間的な負荷分散の実現である.
特定のアクセスポイントに利用者が集中することを避け,より空いているエリアへの移動を促すことで,各アクセスポイントの負荷を均等化し,全体としての通信品質を向上させることができる.\\
第二に,各ユーザの通信パターンの最適化である.ユーザによって,必要とする通信帯域は異なるはずであるため,各ユーザが必要とする最低限度の帯域を利用するよう促すことで,全体の帯域利用効率を高めることができる.\\
第三に,ユーザの行動を最小限に抑えることである.ユーザが無駄な移動を行ったり,ランダムに移動したりすれば,ネットワーク全体の負荷が増大し,通信品質が低下する可能性がある.
そのため,ユーザに対して最小限の行動で済むような誘導を行うことで,通信品質の安定化を図ることができる.
さらに,この考え方に基づいたシステムを用いることで,ユーザにかかる負担が最小限度で済むため,継続してシステムを利用してもらえる可能性が高まる.\\
本研究では,これらの行動変容を実現するための具体的な仕組みとして,利用者へのリアルタイムなフィードバックや推奨行動の提示,ゲーミフィケーション要素の導入,インセンティブ設計などを検討する.
また,提案手法の実用性を評価するため,実環境またはシミュレーション環境において実験を行い,通信品質の改善効果を定量的に測定する.\\
最終的に本研究は,ネットワーク管理における新たなパラダイムとして,「人間中心型ネットワーク管理」の概念を提唱し,その実現可能性と有効性を示すことを目指す.
技術と人間の協調によって,より快適で持続可能な通信環境を構築するための知見を提供することが,本研究の最終的な到達目標である.

\subsection{本論文の構成}
本論文の構成を以下に示す.
第2章では,本研究に関連する背景知識について述べる.
第3章では,提案手法・システムについて述べる.
第4章では,評価実験について述べる.
第5章で本論文をまとめる.
