%%%%%%%%%% 概要 %%%%%%%%%%
\begin{abstract}
本論文では、エンタープライズ無線LAN環境における通信品質改善のためのユーザ行動支援手法に関する研究について述べる.
近年,インターネット環境の質は,業務の効率や生産性,さらには顧客満足度やサービスの信頼性にまで大きな影響を及ぼす要因となっている.%\cite{buffalo727WiFi}
快適なインターネット環境の整備は,事業や活動の成果に直結する重要な要素である.
企業や大学における,多数のユーザが同時にアクセスするエンタープライズ無線LAN環境 (以下,無線LAN) においても,利用者に対して高品質な通信体験を提供することが求められている.\\

無線LANでは,多数のAPを広範囲にわたって配置し,ユーザの移動や利用状況に応じて柔軟にネットワークを提供する必要がある.
このためには,通信の品質を維持・向上させるための高度な管理が不可欠である.

従来,無線LAN環境の最適化においては,ネットワーク管理者側が,AP設置位置の最適化,チャネルの割り当て調整,通信トラフィックの分散制御などを行ってきた.
このような管理者による調整で無線LAN環境をある程度改善できる一方で,依然として不特定要素は多い.
建物の構造による電波の遮蔽や反射,無線干渉,ユーザの予測困難な移動や利用行動といった,管理者の制御がおよばない要素が多数存在する.
これらの要素が複雑に絡み合うことで,利用者の環境では通信品質のばらつきや一時的な接続不良といった通信不良が発生している.
こうした現状を踏まえた上で,より快適な無線LAN環境を実現するため,従来の管理者主体のアプローチに加え,利用者側の行動を踏まえた無線LAN環境改善のための新たなアプローチが必要とされている.
%利用者端末から通信環境の状況を動的に把握・分析し,改善へとつなげる新たなアプローチが必要である.

本研究では,エンタープライズ向け無線LAN環境における通信品質改善のためのユーザ行動支援手法を提案する.
シミュレーションツールns-3を使用したシミュレーションの結果,【主要な成果】が確認できた.
\end{abstract}
