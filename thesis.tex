\documentclass{ujarticle}

\usepackage{geometry}
\usepackage[dvipdfmx]{graphicx}
\usepackage[dvipdfmx]{color}
\usepackage[dvipdfmx]{hyperref}
\geometry{truedimen, mag=1200} %フォントサイズ指定(1000=10pt)
\geometry{top=42mm, bottom=20mm, left=21mm, right=21mm, includefoot} %用紙設定
\usepackage{comment}                      %\begin{comment}~\end{comment}で複数行コメント
\usepackage{amsmath, amssymb}
\usepackage{graphicx, color}
\usepackage{float}
\usepackage{wrapfig}
\usepackage{booktabs}
\usepackage{subfigure}

\usepackage{ascmac}
\usepackage{here}
\usepackage{txfonts}
\usepackage{listings, jlisting}
\usepackage{url}
\usepackage{paralist}
\usepackage[inline]{enumitem}
\usepackage{caption}

\renewcommand{\lstlistingname}{リスト}
\def\doi#1{\normalfont DOI: #1}
\lstset{language=c,
  basicstyle=\ttfamily\scriptsize,
  commentstyle=\textit,
  classoffset=1,
  keywordstyle=\bfseries,
  frame=tRBl,
  framesep=5pt,
  showstringspaces=false,
  numbers=left,
  stepnumber=1,
  numberstyle=\tiny,
  tabsize=2,
  breaklines=true,
}

% キャプション文字を小さめに
\captionsetup{font=small}

% ipsjunsrt.bstを使うための設定
\makeatletter
\def\urlj{%
  \@ifnextchar[{\n@urlj}{\@urlj}%
}

\def\n@urlj[#1]#2{%
  \normalfont
  \if #11\relax
    入手先\\
  \else
  \fi
  \normalfont $\langle$\nobreak#2\nobreak$\rangle$%
}

\def\@urlj#1{%
  \normalfont 入手先$\langle$\nobreak#1\nobreak$\rangle$%
}

\def\urle{%
  \@ifnextchar[{\n@urle}{\@urle}%
}

\def\n@urle[#1]#2{%
  \normalfont
  \if #11\relax
    available \\ from
  \else
  \fi
  $\langle$\nobreak#2\nobreak$\rangle$%
}

\def\@urle#1{%
  \normalfont available from $\langle$\nobreak#1\nobreak$\rangle$%
}

\def\refdatej#1{\normalfont (参照{#1})\<}
\def\refdatee#1{\normalfont (accessed {#1})}
\def\doi#1{\normalfont DOI: #1}

\def\acknowledgment{\par
  \ifDS@english
    {\bfseries{Acknowledgments}}\hskip1em\ignorespaces
  \else
    {\bfseries{謝辞}}\hskip1\zw\ignorespaces
  \fi
}
\def\:{:}
\makeatother

%%% Sectionの設定 %%%
\makeatletter
\renewcommand{\section} {\@startsection{section}{1}{\z@} %
    {.9\Cvs \@plus.5\Cdp \@minus.2\Cdp} %
    {.5\Cvs \@plus.3\Cdp} %
    {\reset@font\Large\bfseries}}

\renewcommand{\subsection}{\@startsection{subsection}{2}{\z@} %
    {.7\Cvs \@plus.5\Cdp \@minus.2\Cdp} %
    {.3\Cvs \@plus.3\Cdp} %
    {\reset@font\normalsize\bfseries}}

\renewcommand{\subsubsection}{\@startsection{subsubsection}{3}{\z@} %
    {0.5\Cvs \@plus.5\Cdp \@minus.2\Cdp}  % original: 1.5
    {0.2\Cvs \@plus.3\Cdp} % 0.5
    {\reset@font\normalsize\bfseries}}

\newcommand{\Section}[1]{\section*{#1}
\addcontentsline{toc}{section}{#1}}
\makeatother

\begin{document}

%%%%%%%%%% 表紙 %%%%%%%%%%
\thispagestyle{empty}   %表紙のみページ番号なし
\setcounter{page}{0}    %表紙を0ページ目に設定

\begin{center}
{\Large
  令和7年度 卒業論文
} \\
\vspace{9mm}
{\LARGE
 エンタープライズ無線LAN環境における通信品質改善のためのユーザ行動支援手法
} \\
\end{center}

\vspace{30mm}
\begin{table}[hb]
\begin{center}
{\Large
\begin{tabular*}{12.5cm}{p{6.25cm}p{6.25cm}}
  &\rule[0mm]{0mm}{0mm} \\
  \multicolumn{2}{c}{電気通信大学 情報理工学域} \\
  \multicolumn{2}{c}{I類 コンピュータサイエンスプログラム} \\
  \multicolumn{1}{r}{学籍番号} & 2210182 \rule[0mm]{0mm}{6.5mm}\\
  \multicolumn{1}{r}{氏名} & 上川雅弘 \\
  \multicolumn{1}{r}{指導教員} & 矢﨑 俊志\\
  \multicolumn{2}{c}{令和8年1月30日} \rule[0mm]{0mm}{6.5mm}\\
\end{tabular*}
}
\end{center}
\end{table}

\newpage
%%%%%%%%%% 概要 %%%%%%%%%%
%%%%%%%%%% 概要 %%%%%%%%%%
\begin{abstract}
本論文では、エンタープライズ無線LAN環境における通信品質改善のためのユーザ行動支援手法に関する研究について述べる.
近年,インターネット環境の質は,業務の効率や生産性,さらには顧客満足度やサービスの信頼性にまで大きな影響を及ぼす要因となっている.%\cite{buffalo727WiFi}
快適なインターネット環境の整備は,事業や活動の成果に直結する重要な要素である.
企業や大学における,多数のユーザが同時にアクセスするエンタープライズ無線LAN環境 (以下,無線LAN) においても,利用者に対して高品質な通信体験を提供することが求められている.\\

無線LANでは,多数のAPを広範囲にわたって配置し,ユーザの移動や利用状況に応じて柔軟にネットワークを提供する必要がある.
このためには,通信の品質を維持・向上させるための高度な管理が不可欠である.

従来,無線LAN環境の最適化においては,ネットワーク管理者側が,AP設置位置の最適化,チャネルの割り当て調整,通信トラフィックの分散制御などを行ってきた.
このような管理者による調整で無線LAN環境をある程度改善できる一方で,依然として不特定要素は多い.
建物の構造による電波の遮蔽や反射,無線干渉,ユーザの予測困難な移動や利用行動といった,管理者の制御がおよばない要素が多数存在する.
これらの要素が複雑に絡み合うことで,利用者の環境では通信品質のばらつきや一時的な接続不良といった通信不良が発生している.
こうした現状を踏まえた上で,より快適な無線LAN環境を実現するため,従来の管理者主体のアプローチに加え,利用者側の行動を踏まえた無線LAN環境改善のための新たなアプローチが必要とされている.
%利用者端末から通信環境の状況を動的に把握・分析し,改善へとつなげる新たなアプローチが必要である.

本研究では,エンタープライズ向け無線LAN環境における通信品質改善のためのユーザ行動支援手法を提案する.
シミュレーションツールns-3を使用したシミュレーションの結果,【主要な成果】が確認できた.
\end{abstract}


\newpage


%%%%%%%%%% 目次 %%%%%%%%%%
\tableofcontents
\clearpage

%%%%%%%%%% はじめに %%%%%%%%%%
\section{はじめに}

\subsection{研究の背景}
現代社会において,インターネットは電気や水道と同様に,日常生活や社会活動を支える重要な基盤となっている.
スマートフォンやタブレット,ノートパソコンなどのモバイル端末の普及により,人々がインターネットへ接続する機会は大幅に増加している.
特に,公共空間における無線LAN(Wi-Fi)環境の整備は急速に進展しており,仕事,学習,娯楽,情報収集など,あらゆる場面でネットワーク接続が前提となっている.
特に,カフェやレストラン,図書館,商業施設,イベント会場,空港などの公共空間においては,無線LAN環境の有無やその使いやすさが利用者にとって重要な評価基準の一つとなっている.
観光庁の調査(2025)\cite{sightseeing-report-2024}によれば,訪日外国人が旅行中に不便を感じた項目の一つに「無料公衆無線LAN環境の不足」が挙げられており,
空港や商業施設における環境整備は,単なる利便性向上に留まらず,施設全体の評価に直結する重要な要素となっている.
そのため,Wi-Fi設備は多くの施設において,必要な設備として認識されるようになっている.\\

このような背景のもと,多くの施設運営者は無線LAN環境の整備に積極的に取り組んできた.
具体的には,アクセスポイントの増設,高性能な通信機器の導入,より広帯域なインターネット回線の契約など,設備面での投資が進められている\cite{MIRAITONE2022}.
しかし,こうした設備投資にもかかわらず,実際の利用現場では「Wi-Fiが繋がりにくい」「通信速度が遅い」「接続が頻繁に切断される」といった不満が利用者から寄せられることもある.
特に,多数の利用者が同時にネットワークを利用する状況,例えば大規模なイベント会場や講演会場,混雑時のカフェなどにおいては,通信品質の低下が顕著に現れることが多い.\\

現在,これらの問題に対応するため,ネットワーク管理者はさまざまな技術的対策を講じている.\\
帯域制御による通信量の調整,QoS(Quality of Service)設定による通信の優先制御,チャネル割り当ての最適化,
電波干渉の原因となる機器の特定や除去,アクセスポイント間での動的な負荷分散などがその代表例である\cite{Cisco2015}.
さらに,トラフィックの常時監視やログ解析,異常検知システムの導入など,運用管理の自動化や効率化も進められてきた.
しかしながら,これらの管理者側の取り組みにもかかわらず,通信品質の問題が完全に解消されているとは言い難い.\\

その要因の一つとして,従来の対策が主にネットワーク機器やインフラ側の制御に注目しており,
実際にネットワークを利用するユーザの行動が十分に考慮されてこなかった点が挙げられる.
例えば,特定の場所に利用者が集中することによるアクセスポイントへの負荷の偏り,バックグラウンドで動作するアプリケーションによる不要な通信,
複数の利用者が同時に大容量データの通信を行うことによる混雑など,利用者の行動はネットワーク全体の通信品質に大きな影響を与えている.
これらの問題を単に技術的な課題として捉えるのではなく,通信システムを構成する重要な要素として「利用者」を位置づけ,
利用者の行動も含めて通信環境全体を改善する視点が必要であると考えられる.


\subsection{研究の目的}
本研究の目的は,従来のネットワーク管理手法に加えて,
利用者の行動に着目することで,エンタープライズ無線LAN環境における通信品質の改善を実現することである.
すなわち,人間を通信環境の単なる利用者としてではなく,通信品質に影響を与える存在として捉え,
利用者に対して適切な行動を促し,通信の品質を改善することを目標とする.\\

これまでの通信品質改善手法は,ネットワーク機器の高性能化や設定の最適化といった,管理者側による技術的対策に大きく依存してきた.
しかし,いかに高性能な機器を導入したとしても,利用者が特定のアクセスポイント周辺に集中したり,必要以上に通信帯域を消費したりする状況では,期待される通信品質を維持することは難しい.
このような問題意識から,本研究では利用者の行動を通信環境改善の一要素として積極的に取り入れる.\\

具体的には,利用者の行動を適切に誘導することで,
以下の三つの効果を目指す.\\

第一に,空間的な負荷分散である.\\
利用者が特定のアクセスポイントに集中することを避け,比較的空いているエリアへの移動を促すことで,
各アクセスポイントの負荷を均等にし,全体として安定した通信品質を実現する.\\

第二に,各利用者の通信利用の効率化である.
利用者ごとに必要とする通信量は異なるため,
各利用者が自身の要求を満たすために必要十分な通信帯域を利用するよう促すことで,ネットワーク全体の帯域利用効率の向上を図る.\\

第三に,利用者の行動負担の軽減である.
無計画な移動や頻繁な場所の変更は,ネットワークの状態を不安定にする可能性がある.
そのため,利用者に対して最小限の移動や操作で済むような行動を提示することで,通信品質の安定化と利用者の負担軽減の両立を目指す.
利用者の負担が小さい手法であれば,継続的に利用されやすいという利点も期待できる.\\

本研究では,これらの行動変化を実現するための方法として,利用者への通信状況の提示や推奨行動の通知などの仕組みを想定する.
そして,提案手法の有効性を検証するため,シミュレーション環境を用いた評価を行い,通信品質の変化を定量的に比較・分析する.
本研究を通じて,利用者の行動を考慮した通信環境改善手法が,エンタープライズ無線LAN環境において有効であることを示す.

\subsection{本論文の構成}
本論文の構成を以下に示す.
第2章では,本研究に関連する背景知識について述べる.
第3章では,提案手法・システムについて述べる.
第4章では,評価実験について述べる.
第5章で本論文をまとめる.

\clearpage
%%%%%%%%%% 背景知識 %%%%%%%%%%
%%%%%%%%%% 背景知識 %%%%%%%%%%
\section{背景知識}
\subsection{無線LANの基礎技術\cite{nttwest2024},\cite{Wifistandard}}
無線LAN(Wireless Local Area Network)は, IEEE 802.11標準規格に基づく無線通信技術により構築されるネットワークである.
システムは主にアクセスポイント(AP: Access Point)とクライアント端末から構成される.
APが2.4GHz帯または5GHz帯(IEEE 802.11ax以降では6GHz帯も利用可能)の電波を用いてクライアントとの双方向通信を実現する.
近年の規格発展により, IEEE 802.11ac(Wi-Fi 5)では最大6.9Gbps, IEEE 802.11ax(Wi-Fi 6)では最大9.6Gbpsの理論スループットが達成されている.
しかし, 実環境では後述する様々な要因により, 実効スループットは理論値を大きく下回ることが知られている.
\subsection{無線LANの通信方式\cite{Wifistandard}}
無線LANは, Carrier Sense Multiple Access with Collision Avoidance(CSMA/CA)方式を採用している.
この方式では, 端末が通信を開始する前にチャネルの状態を監視し, 他の端末が通信中でないことを確認してから送信を行う.
もしチャネルが使用中であれば, ランダムなバックオフ時間を待機した後に再度チャネルの状態を確認する.
これにより, 複数の端末が同時に通信を試みた際の衝突を回避し, 効率的なチャネル利用を実現している.
さらに, Acknowledgment(ACK)フレームを用いた確認応答機能により, 送信データの正確な受信を保証している.
受信側は正常にデータを受信した場合にACKフレームを送信し, 送信側はこれを受信することでデータの到達を確認する.
ACKが受信されない場合, 送信側は再送を行うことで信頼性の高い通信を実現している.
\subsection{無線LANのチャネルと周波数帯域\cite{Wifistandard}}
無線LANは, 2.4GHz帯と5GHz帯,6GHz帯の周波数帯域を利用して通信を行う.
2.4GHz帯は, 電波の到達距離が長く, 障害物に対する透過性が高い一方で, 他の無線機器(電子レンジ,Bluetooth機器など)との干渉が発生しやすい\cite{FURUNO2024}.%参考文献
一方, 5GHz帯は, より広い帯域幅を提供し, 高速な通信が可能であるが, 電波の到達距離が短く, 障害物による減衰が大きいという特性がある.
無線LANでは, 複数のチャネルが定義されており, 各チャネルは特定の周波数範囲を占有する.
2.4GHz帯では, 通常1~14チャネルが利用可能であり, 各チャネルは20MHzの帯域幅を持つ.
5GHz帯では, より多くのチャネルが利用可能であり, 20MHz, 40MHz, 80MHz, 160MHzの帯域幅を持つチャネルが存在する.
チャネル選択は, 干渉の回避と通信品質の最適化において重要な役割を果たす.

\subsection{無線LANにおける通信品質のパラメータ\cite{Wifistandard}}
無線LAN環境における通信品質は, 以下の主要なパラメータが存在する.
\begin{itemize}
    \item スループット (Throughput) \\
    単位時間あたりに実際に転送可能なデータ量(bps)を指す.
    理論上の最大通信速度(帯域幅)とは異なり,プロトコルのオーバーヘッドやネットワークの混雑状況を加味した実効速度である.
    この値の低下は,ファイル転送時間の増大や動画の低画質化など,ユーザ体感品質(QoE:Quality of Experience)の低下に直結する.

    \item 遅延(レイテンシ / Latency) \\
    パケットが送信元から宛先へ到達するまでに要する時間(ms).
    双方向通信においては往復時間(RTT: Round Trip Time)で評価される.
    リアルタイム通信において極めて重要であり,この値が増大すると,Web会議の音声途切れやオンラインゲームの操作遅延など,即時性を要するアプリケーションの品質を損なう.

    \item パケット損失率 (Packet Loss Rate)\\
    送信されたパケットが,輻輳(混雑)やノイズ,伝送路の不安定さによって正常に受信されない割合.
    損失が発生すると,TCP等のプロトコルによる再送制御が作動し,結果として遅延の増大や実効スループットの低下を招く.

    \item 受信信号強度 (RSSI: Received Signal Strength Indicator) \\
    受信した電波の強度を表す指標(dBm).値が0に近いほど信号が強く,通信可能範囲や接続の安定性を判断する物理的な指標となる.
    ただし,強度が十分であっても後述するSNRが低い場合は,安定した通信が困難になることがある.

    \item 信号対雑音比 (SNR: Signal-to-Noise Ratio) \\
    所望の信号電力と背景雑音電力の比(dB).
    SNRが高いほど信号がノイズに埋もれずクリアであることを示し,結果としてスループットが向上する.

    \item チャネル使用率 (Channel Utilization) \\
    特定の周波数チャネルが単位時間のうちに占有されている時間の割合(\%).
    無線LAN等の共有媒体においては,他端末や干渉波による使用率が高いほどスループットの低下か遅延の原因となる.
\end{itemize}

\subsection{チャネル使用率}
チャネル使用率は, 無線LANチャネルが占有されている割合を示す指標であり, 通信品質に大きな影響を与える.
チャネル使用率は, 他の端末による通信が頻繁に発生している時や,外乱・電波干渉が起こっている時などに高くなり,新たな通信の開始が困難になる.
これにより, スループットの低下, 遅延の増加, パケットロス率の上昇といった通信品質の劣化が生じる.
チャネル使用率は, 通常0\%から100\%の範囲で表され, 100\%に近い値はチャネルがほぼ常に占有されている状態を示す.\\
チャネル使用率の測定は, 無線LAN機器や専用の解析ツールを用いて行われる.
これらのツールは, チャネルの占有時間を監視し, 使用率をリアルタイムで算出する機能を備えている.
ネットワーク管理者は, チャネル使用率の情報を活用して, チャネルの最適化や負荷分散の施策を講じることが可能である.
チャネル使用率$U_{ch}$は,チャネル使用状況を測定できた時間のうち,実際にチャネルが使用されていた時間の割合として定義される.


\subsection{AP密集環境における干渉問題}
エンタープライズ無線LAN環境では,通信容量の確保やカバレッジの向上を目的として,同一環境内に多数のAPが配置される.
しかし,APが高密度に配置される環境では,隣接チャネル干渉や同一チャネル干渉が発生しやすくなり,通信品質の低下を招く要因となる.

同一チャネル干渉は,複数のAPが同一のチャネルを使用することにより発生し,CSMA/CA方式に基づく送信待ち時間の増加を引き起こす.
一方,隣接チャネル干渉は,周波数帯域が部分的に重複するチャネルを使用する場合に発生し,パケット誤り率の増加や再送の増大につながる.
これらの干渉は,物理的な距離が十分に離れていないAP間で顕著となり,結果としてスループットや遅延特性に悪影響を及ぼす.


\subsection{管理外APによる影響}
近年の無線LAN環境では,管理者が設置・管理していない通信機器が通信品質に影響を与えるケースが増加している.
具体的には,来訪者が持ち込むモバイルルータやスマートフォンのテザリング機能などが挙げられる.
これらの管理外APは,既存の無線LANシステムと同一または隣接するチャネルを使用する場合が多く,予期せぬチャネル使用率の増加を引き起こす.

管理外APはネットワーク管理者からは直接制御することができず,その存在や通信状況を正確に把握することが困難である.
そのため,APと端末間の距離や受信信号強度のみを基準とした接続制御では,実際の通信品質を適切に評価できない場合がある.

\subsection{AP選択方式とその課題}
無線LANにおけるAP選択は,クライアント端末がどのAPに接続するかを決定する重要なプロセスである.
一般的な端末では,受信信号強度(RSSI)が最も高いAPを選択する方式が広く用いられている.
この方式は実装が容易である一方で,APの混雑状況やチャネル使用率を考慮していないという課題がある.

その結果,多数の端末が特定のAPに集中し,通信品質の低下や負荷の偏りが発生することが知られている.
このような問題は,APの設置間隔が短いエンタープライズ環境において特に顕著であり,RSSIのみを用いたAP選択の限界を示している.

\subsection{ユーザ移動と通信品質の関係}
無線LAN環境における通信品質は,ユーザ端末の位置に大きく依存する.
ユーザがAPに近づくことで受信信号強度やSNRが向上し,高速な変調方式の利用が可能となる.
一方で,AP周辺は多くのユーザが集中しやすく,チャネル使用率の増加によって必ずしも高いスループットが得られるとは限らない.

また,ユーザのわずかな移動によっても,接続APや利用チャネルが変化し,通信品質が大きく変動する場合がある.
このことから,ユーザの位置情報と無線環境情報を組み合わせて評価することが,通信品質の最適化において重要であるといえる.

\subsection{無線LANにおける負荷分散の考え方}
無線LAN環境では,特定のAPやチャネルに通信負荷が集中することを避けるため,負荷分散が重要な課題となる.
従来の負荷分散手法は,主にAP側での送信制御や接続制限といった管理者主導の手法が中心であった.
しかし,これらの手法はリアルタイム性や利用者体感品質の観点から限界がある.

近年では,利用者の行動や接続選択がネットワーク全体の性能に影響を与えるという観点から,ユーザを含めた負荷分散の重要性が指摘されている.
このような背景から,ネットワーク側の情報を活用してユーザに適切な行動を促す手法が注目されている.

\subsection{シミュレーションツール ns-3\cite{ns-3}}

ns-3 は離散事象ネットワークシミュレータであり,インターネットシステム全般の挙動を模擬するためのオープンソースソフトウェアである.
本ツールは主に研究および教育目的で設計されており,GNU GPLv2 ライセンスの下で公開されている.
ns-3 の基本的な考え方は,実際のネットワークで発生する出来事(イベント)を順序立てて処理し,応答や遅延,通信プロトコルのふるまいを詳細に再現する点にある.

具体的には,ns-3 はネットワークノード,プロトコルスタック,通信チャネルといった要素をソフトウェア内部に構築し,イベントスケジューラを用いて各イベントを処理する.
また,C++ を主言語として使用し,Python バインディングを通じてスクリプト形式でも操作可能な柔軟性を持つ.

Wi-Fi のシミュレーションに関しては,ns-3 が IEEE 802.11 準拠のモデルを標準で提供しているため,
アクセスポイントと端末間の通信や,基本的な物理層・MAC 層の挙動を模擬できる.
具体的には,インフラストラクチャーモードおよびアドホックモードの双方がサポートされ,802.11a/b/g/n などの物理層仕様,伝搬損失モデル,EDCA による QoS 拡張などを設定可能である.

このような特性から,Wi-Fi ネットワークの性能評価やプロトコル設計,トラフィック挙動の解析といった研究課題において,ns-3 は非常に有用なツールである.
実ネットワークを構築せずに多様なシナリオを再現できる点は,コストや時間の節約にもつながる.
\clearpage
%%%%%%%%%%関連研究%%%%%%%%%%%
\section{関連研究}
無線LAN環境の改善に向けた取り組みは数多く報告されている.
Ranasingheら\cite{Ranasinghe2021}やWu\cite{Wu2024}らは,各APに接続される端末数のロードバランシングに関する手法を提案している.
ロードバランシングは無線LAN環境を改善するために管理者側で行う対策としては基本的なものである.
各APに接続される端末の台数をできるだけ均一にすることで,APごとの通信負荷を分散し,特定のAPに過度な負荷が集中することを防ぐ.
一方でユーザの位置的分布や行動特性を変化させるものではないため,実環境における通信品質の改善効果には限界がある.
%主としてAP配置に焦点を当てており,ユーザの行動特性や端末側の視点を考慮したアプローチについては十分に検討されていない.
既製品としても,ユーザの通信に関するQoEを評価・可視化する機能を持つものはあるが,具体的に通信環境を改善する機能としては,AP配置やチャネル割り当ての最適化にとどまっている.

Rowdenらの研究では,電波強度をVR技術を用いて可視化し,利用者が最適なAPを選択可能とする手法が提案されている\cite{Rowden2023}.
この手法はVRゴーグル等の専用機器を必要とするため,実環境で利用者に提供する機能としては現実的ではない.
現実的な運用環境においては,利用者が自身の端末のみで通信環境を改善するための行動を確認できることが重要な要件となる.

本研究が目指すように,利用者に行動を促すことで,無線LAN環境を改善する取り組みもある\cite{Miyata2012}.
Miyataらは,利用者の協調的な移動を考慮した新たなAP選択手法を提案している.
同研究は,各利用者がそれぞれ指定する「移動可能距離」と「許容スループット」を条件とし,無線LANシステム全体のスループットを最大化するための移動を利用者に促すものである.
この手法は,スループットを評価指標として,APと利用者の距離のみに着目した手法であり,電波干渉や物理的障害物,他利用者数の動的変化といった要因は考慮されていない.
また,シミュレーションのみで手法を評価しており,実環境における適用についてはその効果は未解明である.
この手法はAPや利用者の位置を特定する方法としてAPとは異なるセンサーを多数設置することを想定しており,実環境においてはコスト面での制約が大きい点も課題である.

一方で,無線LAN環境の状態を適切に把握し,ユーザへの行動支援に活用するためには,通信品質を定量的に評価する指標が必要となる.
この点において,チャネル利用率は無線LANの混雑状況や通信品質を直接的に反映する有用な指標である.
チャネル利用率に関連する研究にはBianchiらによるモデルがある\cite{Bianchi2000}.
この研究では理想的な環境を定義した上で,無線LANにかかる処理を定式化することで,チャネル利用率に関連するようなパラメータを解析的に求めている.
%
より直接的なチャネル利用率の推定としては,Zhaoらによるキャリアセンスに着目した手法がある\cite{Zhao2013}.
Zhaoらはチャネルの利用されていないIDLE時間を新たなセンシング手法で推定し,実効的な最大スループットを求める手法を提案した.
この手法は他手法に比べて,より実験値に近い値を算出できていることが示されている.
%
端末の情報のみを用いた無線環境の推定としては,Rossiらによる実験がある\cite{Rossi2014}.
Rossiらは特殊機材などを用いず,Linux 向けカスタムドライバを導入することでのみ得られる情報を用いて,無線LAN環境の推定を行っている.
%
Remote Access Network(RAN)分野でも伝送路の状態推定が行われており,Guttermanら\cite{Gutterman2019}による研究がある.
RANスライシングに代表される5Gネットワークでは,適切なサービス提供を行うため,伝送路の状態とユーザの需要を適切に識別・推定したうえでスケジューリングを行う必要がある.
この課題に対して,Guttermanらは古典的な統計時系列モデルと機械学習であるLSTMを組み合わせた手法を提案している.
\clearpage
%%%%%%%%%% 実装に関して %%%%%%%%%%
%%%%%%%%%% 実装に関して %%%%%%%%%%
\section{提案手法}

\subsection{提案手法の概要}
本研究では,エンタープライズ無線LAN環境における通信品質の改善を目的として,ネットワーク管理側の制御のみに依存するのではなく,利用者自身の能動的な行動変容を促すことにより,通信環境全体の最適化を図るユーザ行動支援手法を提案する.
近年,オフィスビルや大学キャンパスなどのエンタープライズ環境では,無線LANが業務や学習を支える基盤インフラとして広く利用されている一方で,利用者数や端末数の増加に伴い,通信品質のばらつきやスループット低下が問題となっている.

従来の無線LAN管理における最適化アプローチは,主にアクセスポイント(AP)の設置位置の最適化,チャネル割り当ての調整,送信出力制御といった,管理者主導の技術的施策に焦点が当てられてきた.
これらの手法は,静的あるいは準静的な環境を前提とした場合には有効であるものの,実際の運用環境では必ずしも十分な効果を発揮しない場合が多い.

その要因として,建物構造による電波の遮蔽や反射,不特定多数のユーザによる予測困難な移動,さらには来訪者が持ち込むモバイルルータやテザリング端末といった管理外の通信端末による干渉など,時間的・空間的に変動する要素が挙げられる.
これらの動的な要因を,管理者側の制御のみでリアルタイムに把握し,完全に解決することは困難である.

そこで本研究では,通信システムを構成する重要な要素として「ユーザ」に着目し,ネットワーク側が把握可能な環境情報と,ユーザ自身が提供する情報を組み合わせることで,ユーザに対して適切な行動指針を提示する新たなアプローチを検討する.
具体的には,ネットワーク管理側から取得可能な情報に基づき,ユーザに対して最適な接続先APや,通信品質の向上が期待できる移動先を提示することで,高品質な通信体験の実現を目指す.

提案アルゴリズムの特徴を以下にまとめる.
\begin{itemize}
  \item APとユーザ間の距離に加え,チャネル使用率および接続ユーザ数を考慮したAP選択を行う点
  \item ユーザの許容移動距離および最低要求スループットを制約条件として明示的に導入している点
  \item 新規ユーザ単体ではなく,システム全体のスループット最大化を目的としている点
\end{itemize}

これにより,実環境における電波干渉や負荷の偏りを考慮した,現実的かつ合理的なAP選択および移動誘導が可能となる.

\subsection{提案アルゴリズム}

本研究では,エンタープライズ無線LAN環境において新たに接続を行う利用者に対し,通信品質の向上を目的として,最適なアクセスポイント(AP)および移動方向を提示するアルゴリズムを提案する.
提案手法は,APと利用者端末の距離に基づいて接続先を決定する従来手法を拡張し,APが使用しているチャネルの使用率や接続ユーザ数といった無線環境の混雑状況を考慮する点に特徴がある.
本研究では,通信品質(QoE)の評価指標としてスループットを採用し,新規ユーザの接続によってシステム全体のスループットが最大化されることを目的とする.

\subsubsection{入力情報と前提条件}

提案アルゴリズムは,新規ユーザ$u$に対して以下の情報が与えられていることを前提とする.また,無線LAN環境内に存在するAPの集合を$A$とし,各AP $a \in A$に対して以下の情報が既知であると仮定する.
表\ref{tab:input_info}に入力情報の一覧を示す.なお,スループットの平均値は接続ユーザのスループットの調和平均として計算されるものとする.

\begin{table}[h]
\centering
\caption{提案アルゴリズムの入力情報}
\label{tab:input_info}
\begin{tabular}{l|l|l}
\hline
カテゴリ & パラメータ & 説明 \\ \hline \hline
\multirow{3}{*}{ユーザ情報}
  & $P_u$ & 現在位置 \\
  & $d_{th}$ & 許容最大移動距離 \\
  & $\theta_{th}$ & 最低要求スループット \\ \hline
\multirow{4}{*}{AP情報}
  & $P_a$ & APの位置座標 \\
  & $n_a$ & 接続中のユーザ数 \\
  & $U_c$ & 使用チャネルの使用率 \\
  & $\theta^{before}_a$ & 接続前の平均スループット \\ \hline
\end{tabular}
\end{table}

なお,本研究では位置情報の取得方法や入力インタフェースについては議論せず,ユーザが自己申告により現在位置を入力できるものと仮定する.

\subsubsection{接続候補APの抽出}

新規ユーザが移動可能な範囲内に存在するAPのみを接続候補として抽出する.
ユーザの移動は移動ベクトル$m$により表され,移動後のユーザ位置は$P_u - m$とする.
このとき,移動後のユーザとAP $a$との距離$d_{u,a}$は以下の式で定義される.

\begin{equation}
d_{u,a} = \left| P_a - (P_u - m) \right|
\end{equation}

距離$d_{u,a}$が許容最大移動距離$d_{th}$以下となるAPの集合を,接続候補AP集合$A_{cand}$とする.

\subsubsection{接続前後のスループット推定}

次に,各接続候補APに対して,新規ユーザ接続前後のスループットを推定する.
AP $a$に接続されている既存ユーザの平均スループット$\theta^{before}_a$は,接続ユーザのスループットの調和平均として与えられているものとする.

新規ユーザがAP $a$に接続した場合のスループット$\theta^{after}_{a,m}$は,APとユーザ間の距離およびチャネル使用率$U_c$を考慮して推定される.
このとき,新規ユーザのスループットが最低要求スループット$\theta_{th}$を下回る場合,または移動距離が許容最大移動距離$d_{th}$を超える場合には,当該APを接続候補から除外する.

\subsubsection{評価スコアの算出}

残った接続候補APに対して,以下の4つの評価指標を用いて総合スコアを算出する.
\begin{enumerate}
  \item 接続後スループットに基づくスループットスコア
  \item ユーザとAPの距離に基づく距離スコア
  \item APが使用するチャネルの使用率に基づくチャネル使用率スコア
  \item APに接続されているユーザ数に基づく接続ユーザ数スコア
\end{enumerate}

各スコアは正規化された値として算出され,重み係数$w_i$を用いて以下の式により総合スコア$Score(a)$を計算する.

\begin{equation}
Score(a) = \sum_i w_i \cdot score_i(a)
\end{equation}

\subsubsection{最適APおよび移動ベクトルの決定}

提案アルゴリズムでは,総合スコアが最大となるAPを単純に選択するのではなく,総合スコア上位の複数APを候補として扱う.
その上で,各候補APに対して,新規ユーザ接続後のシステム全体のスループットを算出し,これが最大となるAP $a^*$と移動ベクトル$m^*$を最終的な出力とする.

システム全体のスループット$\Theta^{after}_{a,m}$は,以下の式で表される.

\begin{equation}
\Theta^{after}_{a,m}
= \theta^{after}_{a,m} + \sum_{i \neq a} \theta^{before}_i
\end{equation}

以上で述べた提案アルゴリズムを擬似コードとしてAlgorithm\ref{alg:optimalAP}にまとめる.
%
\begin{algorithm}[tbh]
    \begin{algorithmic}[1]
        \State \textbf{Input:}
        $P_u$(ユーザ現在位置),
        $d_{th}$(最大許容移動距離),
        $\theta_{th}$(最低要求スループット),
        $A$(AP集合)

        \State \textbf{Output:}
        $a^*$(選択AP),
        $m^*$(推奨移動ベクトル)

        \Statex
        \Statex \Comment{--- AP候補抽出 ---}
        \State $A_{cand} \gets \{ a \in A \mid \|P_a - P_u\| \le d_{th} \}$

        \Statex
        \Statex \Comment{--- 接続可否判定 ---}
        \For{each $a \in A_{cand}$}
            \State Compute $\theta_a^{before}$ using harmonic mean of existing users
            \State Estimate $b_{new,a}$ from RSSI at $P_u$
            \State Compute $\theta_{a,0}^{after}$ assuming no movement
            \State Compute $\theta_{new,a,0}$ for the new user
            \If{$\theta_{new,a,0} < \theta_{th}$}
                \State $A_{cand} \gets A_{cand} \setminus \{a\}$
            \EndIf
        \EndFor

        \Statex
        \Statex \Comment{--- スコア計算 ---}
        \For{each $a \in A_{cand}$}
            \State Compute $score_{\theta_a}$ from $\theta_{a,0}^{after}$
            \State Compute $score_{d_{u,a}}$ from $\|P_a - P_u\|$
            \State Compute $score_{U_c}$ from channel utilization
            \State Compute $score_{n_a}$ from number of users
            \State $Score(a) \gets \sum_i w_i \cdot score_i(a)$
        \EndFor

        \Statex
        \Statex \Comment{--- 最終選択 ---}
        \State $A_{cand}' \gets$ top-3 APs by $Score(a)$
        \State $\{a^*, m^*\} \gets
        \arg\max_{a \in A_{cand}',\, \|m\|\le d_{th}}
        \theta_{a,m}^{after}$

        \State \Return $\{a^*, m^*\}$
    \end{algorithmic}
    \caption{提案アルゴリズム}
    \label{alg:optimalAP}
\end{algorithm}


\subsection{チャネル使用率の推定}

提案アルゴリズムでは,AP選択の評価指標の一つとしてチャネル使用率を用いる.
2.5節で述べた通り,無線LAN環境において,チャネル使用率はAPがどの程度通信に使用されているかを表す重要な指標であり,通信品質やスループットの低下を把握する上で不可欠である.
802.11k 規格では,チャネル使用率以外にもいくつかの無線資源管理に関する情報が定義されており,無線LAN クライアントは,これらの情報を元に最適なAPを選択することができる.\\
しかしながら,一部の無線LAN 環境においては,AP の機能・性能やネットワーク構成の制約などの理由から,このチャネル使用率に関する情報が提供されない場合や,取得が困難な場合がある.
またエンドユーザーの端末といった改修が憚られる端末においては,特別なドライバなどが導入できない場合も多い.\\

本節では,シミュレーション環境における評価を想定し,RSSI,ユーザの通信帯域幅,およびAP接続ユーザ数といった容易に取得可能なパラメータを用いて,チャネル使用率を推定する手法を提案する.

\subsubsection{推定手法の方針}

チャネル使用率は,無線チャネルが実際に通信に使用されている時間の割合を表す指標であり,以下の要因によって変動する.
\begin{itemize}
  \item 接続ユーザ数:ユーザ数が多いほど,チャネルへのアクセス競合が増加し,使用率が上昇する
  \item 通信帯域幅:各ユーザが要求する帯域幅が大きいほど,チャネル占有時間が増加する
  \item 伝搬環境:APとユーザ間の距離や障害物により,RSSI(受信信号強度)が変化し,これが再送制御やMCS(Modulation and Coding Scheme)選択に影響を与える
\end{itemize}

これらの要因を踏まえ,本研究ではRSSI,通信帯域幅,ユーザ数の3つのパラメータを説明変数として,チャネル使用率を目的変数とする回帰モデルを構築する.

\subsubsection{データ収集と回帰分析}

チャネル使用率推定モデルの構築には,以下の手順を用いた.

\paragraph{シミュレーション条件の設定}
ネットワークシミュレータを用いて,ユーザ数,各ユーザの通信帯域幅,およびAPとユーザ間の距離をパラメータとして変化させた複数のシミュレーションを実施した.
具体的には,以下の条件を組み合わせた実験を行った.
\begin{itemize}
  \item ユーザ数:$n = \{1, 2, 3, \ldots, N_{max}\}$
  \item 通信帯域幅:$B = \{B_1, B_2, \ldots, B_k\}$ [Mbps]
  \item APとユーザ間距離:$d = \{d_1, d_2, \ldots, d_l\}$ [m]
\end{itemize}

各条件の組み合わせについて,シミュレーション実行中に真のチャネル使用率$U_c$を測定し,同時にRSSI値を記録した.

\paragraph{回帰モデルの導出}
収集したデータセット$\{(RSSI_i, B_i, n_i, U_{c,i})\}_{i=1}^{M}$に対して,最小二乗法による多変量回帰分析を実施した.
チャネル使用率$\hat{U}_c$を推定する回帰式は,以下の形式で表される.

\begin{equation}
\hat{U}_c = \beta_0 + \beta_1 \cdot RSSI + \beta_2 \cdot B + \beta_3 \cdot n + \epsilon
\end{equation}

ここで,$\beta_0, \beta_1, \beta_2, \beta_3$は回帰係数,$\epsilon$は誤差項である.
回帰係数は,観測データに対する残差平方和を最小化するように決定される.

必要に応じて,パラメータ間の相互作用項や非線形項(例:$RSSI^2$,$n \cdot B$)を導入することで,推定精度の向上を図ることも可能である.

\paragraph{推定式の検証}
導出した回帰式の妥当性を評価するため,訓練データとは異なるテストデータに対して推定を行い,真の値との決定係数$R^2$や平均二乗誤差(RMSE)を算出した.
推定精度が十分に高いことを確認した上で,シミュレーション評価において本推定式を適用する.

\subsubsection{シミュレーションへの適用}

提案手法の性能評価を行うシミュレーションにおいては,各時刻における各APのチャネル使用率を,導出した回帰式を用いてリアルタイムに推定する.
具体的には,シミュレーション内で観測可能なRSSI,各ユーザの通信帯域幅,およびAP接続ユーザ数を入力として,推定式$\hat{U}_c$によりチャネル使用率を算出する.
この推定値を提案アルゴリズムの評価スコア算出に利用することで,チャネル使用率情報が直接取得できない環境においても,提案手法の有効性を評価することが可能となる.

\subsubsection{推定手法の意義と限界}

本推定手法により,シミュレーション環境においても現実的なチャネル使用率の評価が可能となり,提案アルゴリズムの性能検証をより現実に即した条件で実施できる.
また,実環境への適用を考えた場合においても,BSS Load要素を広報しないAPが存在する環境や,クライアント側からの観測のみでシステムを構築する必要がある場合には,本推定手法が有用となる.

一方で,本推定手法はシミュレーション環境において収集したデータに基づいているため,実環境における電波伝搬特性の複雑さや,干渉源の影響,MACプロトコルの詳細な挙動などを完全には反映できない可能性がある.
実環境への適用にあたっては,実測データを用いた回帰モデルの再学習や,機械学習手法の導入による推定精度の向上が今後の課題として挙げられる.


\clearpage
%%%%%%%%%% 評価 %%%%%%%%%%
%%%%%%%%%% 評価 %%%%%%%%%%
\section{評価}

\section{評価}
本章では,提案手法の有効性を検証するために実施したシミュレーション評価について述べる.
具体的には,ネットワークシミュレータを用いて無線 LAN 環境を再現し,提案手法に基づいてユーザが移動した場合と,ユーザがランダムに移動した場合とを比較した.
シミュレーション結果をもとに,システム全体のスループットおよびユーザの移動距離の観点から評価を行った.

\subsection{評価環境}
評価には,ネットワークシミュレータ ns-3(version 3.35)を用いた.\\
無線 LAN の規格としては,エンタープライズ環境で広く利用されているIEEE 802.11ax(2.4GHz 帯)を想定し,一般的なアクセスポイント(AP)をモデル化した.

シミュレーションは 1 回あたり 30 秒間とし,各条件について 100 回の試行を行った.
ユーザが最低限満たすべきスループットとして,Web 会議が可能な通信品質を想定し,最低許容スループットを 30 Mbps と設定した.

評価では,AP 数およびユーザ数が比較的少ない小規模環境と,AP 数およびユーザ数が多い大規模環境の 2 種類のシナリオを設定した.
各シナリオにおいて,新規ユーザはランダムな位置に出現し,既存ユーザはシミュレーション開始時点で AP に接続されているものとした.

\subsection{評価方法}
評価では,提案手法によって算出された移動先にユーザが移動した場合と,ユーザがランダムな方向に移動した場合とを比較対象とした.
いずれの場合も,ユーザは事前に設定された移動可能距離の範囲内でのみ移動するものとする.

評価指標としては,以下の 2 点を用いた.\\
1 つ目は,新規ユーザが移動する前後におけるシステム全体の合計スループットの改善率である.
これにより,提案手法が無線 LAN システム全体の通信品質向上に寄与しているかを評価する.

2 つ目は,ユーザごとの移動距離である.
提案手法では,ユーザに過度な移動を強いない「消極的移動」を方針としている.
そのため,スループットの改善だけでなく,移動距離がどの程度抑制されているかを併せて評価する.

これらの指標について,ランダム移動と提案手法による移動の結果を比較した.

\subsection{評価結果}
AP 数およびユーザ数が少ない小規模環境においては,
提案手法を適用した場合,ランダム移動と比較して
システム全体のスループット改善率の平均値が約 24.6\% 高い結果となった.
また,提案手法では移動距離が 0.5 m 以下に抑えられるユーザが一定数存在する一方で,
最大移動距離近くまで移動するユーザも確認された.

次に,AP 数およびユーザ数が多い大規模環境において評価を行った.
この環境では,ランダム移動と提案手法との間で,
システム全体のスループット改善率の平均値の差は約 2.5\% となった.
一方で,ユーザごとの移動距離に着目すると,
提案手法ではランダム移動と比較して移動距離が抑えられる傾向が確認された.
特に,提案手法における最小移動距離は,
ランダム移動における平均移動距離と同程度であり,
不要な移動を抑制できていることが示された.

\subsection{考察}
評価結果から,提案手法を適用することで,
ランダムに移動した場合と比較して,
システム全体のスループットを改善できることが確認できた.
特に小規模環境では,提案手法による効果が顕著に現れ,
ユーザ行動を適切に誘導することの有効性が示された.

一方で,大規模環境ではスループット改善率の差が小さくなる結果となった.
これは,ユーザ数の増加により無線資源の競合が激化し,
個々のユーザの移動による改善効果が相対的に小さくなったためであると考えられる.
また,初期位置によっては,最大移動距離まで移動しても要求スループットを満たせないユーザが存在することも確認された.

これらの結果から,提案手法は,ランダムな移動と比較してスループット改善と移動距離抑制の両立が可能である一方,
すべてのユーザに対して十分な通信品質を保証できるわけではないことが明らかとなった.
今後は,移動によっても改善が困難なユーザに対する追加的な対策を検討する必要がある.

\clearpage
% %%%%%%%%%% 考察 %%%%%%%%%%
% \input{sections/6-discuttion.tex}
% \clearpage
%%%%%%%%%% 終わりに %%%%%%%%%%
%%%%%%%%% 終わりに %%%%%%%%%%

\section{おわりに}

\subsection{まとめ}
本研究では,【研究内容の要約】について述べた.

\subsection{今後の課題}
今後の課題として,【今後の課題や展望】が挙げられる.

\newpage
%%%%%%%%%% 謝辞 %%%%%%%%%%
\Section{謝辞} %番号なしセクションにする

本研究を進めるにあたり,研究テーマや方針について多大なるご指導を賜りました矢﨑 俊志先生に心より感謝申し上げます.
また,日頃より支えてくださった家族に対しても,ここに感謝の意を表します.

\clearpage
%%%%%%%%%% 参考文献 %%%%%%%%%%
\addcontentsline{toc}{section}{参考文献}
\bibliographystyle{ipsjunsrt} %番号なしのセクションにする
\bibliography{export} %本やWebなども含める(20,30ぐらい)
\clearpage
%%%%%%%%%% 付録 %%%%%%%%%%
\appendix
\section{付録}

【必要に応じて付録を記述】

\end{document}
