\section{はじめに}

\subsection{研究の背景}
現代社会において,インターネットは電気や水道と同様に,日常生活や社会活動を支える重要な基盤となっている.
スマートフォンやタブレット,ノートパソコンなどのモバイル端末の普及により,人々がインターネットへ接続する機会は大幅に増加している.
特に,公共空間における無線LAN(Wi-Fi)環境の整備は急速に進展しており,仕事,学習,娯楽,情報収集など,あらゆる場面でネットワーク接続が前提となっている.
特に,カフェやレストラン,図書館,商業施設,イベント会場,空港などの公共空間においては,無線LAN環境の有無やその使いやすさが利用者にとって重要な評価基準の一つとなっている.
観光庁の調査(2025)\cite{sightseeing-report-2024}によれば,訪日外国人が旅行中に不便を感じた項目の一つに「無料公衆無線LAN環境の不足」が挙げられており,
空港や商業施設における環境整備は,単なる利便性向上に留まらず,施設全体の評価に直結する重要な要素となっている.
そのため,Wi-Fi設備は多くの施設において,必要な設備として認識されるようになっている.\\

このような背景のもと,多くの施設運営者は無線LAN環境の整備に積極的に取り組んできた.
具体的には,アクセスポイントの増設,高性能な通信機器の導入,より広帯域なインターネット回線の契約など,設備面での投資が進められている\cite{MIRAITONE2022}.
しかし,こうした設備投資にもかかわらず,実際の利用現場では「Wi-Fiが繋がりにくい」「通信速度が遅い」「接続が頻繁に切断される」といった不満が利用者から寄せられることもある.
特に,多数の利用者が同時にネットワークを利用する状況,例えば大規模なイベント会場や講演会場,混雑時のカフェなどにおいては,通信品質の低下が顕著に現れることが多い.\\

現在,これらの問題に対応するため,ネットワーク管理者はさまざまな技術的対策を講じている.\\
帯域制御による通信量の調整,QoS(Quality of Service)設定による通信の優先制御,チャネル割り当ての最適化,
電波干渉の原因となる機器の特定や除去,アクセスポイント間での動的な負荷分散などがその代表例である\cite{Cisco2015}.
さらに,トラフィックの常時監視やログ解析,異常検知システムの導入など,運用管理の自動化や効率化も進められてきた.
しかしながら,これらの管理者側の取り組みにもかかわらず,通信品質の問題が完全に解消されているとは言い難い.\\

その要因の一つとして,従来の対策が主にネットワーク機器やインフラ側の制御に注目しており,
実際にネットワークを利用するユーザの行動が十分に考慮されてこなかった点が挙げられる.
例えば,特定の場所に利用者が集中することによるアクセスポイントへの負荷の偏り,バックグラウンドで動作するアプリケーションによる不要な通信,
複数の利用者が同時に大容量データの通信を行うことによる混雑など,利用者の行動はネットワーク全体の通信品質に大きな影響を与えている.
これらの問題を単に技術的な課題として捉えるのではなく,通信システムを構成する重要な要素として「利用者」を位置づけ,
利用者の行動も含めて通信環境全体を改善する視点が必要であると考えられる.


\subsection{研究の目的}
本研究の目的は,従来のネットワーク管理手法に加えて,
利用者の行動に着目することで,エンタープライズ無線LAN環境における通信品質の改善を実現することである.
すなわち,人間を通信環境の単なる利用者としてではなく,通信品質に影響を与える存在として捉え,
利用者に対して適切な行動を促し,通信の品質を改善することを目標とする.\\

これまでの通信品質改善手法は,ネットワーク機器の高性能化や設定の最適化といった,管理者側による技術的対策に大きく依存してきた.
しかし,いかに高性能な機器を導入したとしても,利用者が特定のアクセスポイント周辺に集中したり,必要以上に通信帯域を消費したりする状況では,期待される通信品質を維持することは難しい.
このような問題意識から,本研究では利用者の行動を通信環境改善の一要素として積極的に取り入れる.\\

具体的には,利用者の行動を適切に誘導することで,
以下の三つの効果を目指す.\\

第一に,空間的な負荷分散である.\\
利用者が特定のアクセスポイントに集中することを避け,比較的空いているエリアへの移動を促すことで,
各アクセスポイントの負荷を均等にし,全体として安定した通信品質を実現する.\\

第二に,各利用者の通信利用の効率化である.
利用者ごとに必要とする通信量は異なるため,
各利用者が自身の要求を満たすために必要十分な通信帯域を利用するよう促すことで,ネットワーク全体の帯域利用効率の向上を図る.\\

第三に,利用者の行動負担の軽減である.
無計画な移動や頻繁な場所の変更は,ネットワークの状態を不安定にする可能性がある.
そのため,利用者に対して最小限の移動や操作で済むような行動を提示することで,通信品質の安定化と利用者の負担軽減の両立を目指す.
利用者の負担が小さい手法であれば,継続的に利用されやすいという利点も期待できる.\\

本研究では,これらの行動変化を実現するための方法として,利用者への通信状況の提示や推奨行動の通知などの仕組みを想定する.
そして,提案手法の有効性を検証するため,シミュレーション環境を用いた評価を行い,通信品質の変化を定量的に比較・分析する.
本研究を通じて,利用者の行動を考慮した通信環境改善手法が,エンタープライズ無線LAN環境において有効であることを示す.

\subsection{本論文の構成}
本論文の構成を以下に示す.
第2章では,本研究に関連する背景知識について述べる.
第3章では,提案手法・システムについて述べる.
第4章では,評価実験について述べる.
第5章で本論文をまとめる.
