%%%%%%%%%% 概要 %%%%%%%%%%
\begin{abstract}
本論文では,エンタープライズ無線LAN環境における通信品質改善を目的とした,ユーザ行動支援手法に関する研究について述べる.
近年,インターネット環境の質は,業務効率や生産性のみならず,顧客満足度やサービスの信頼性にも大きな影響を及ぼす要因となっている.
企業や大学など,多数の利用者が同時にアクセスするエンタープライズ無線LAN環境においては,利用者全員に対して安定した高品質な通信体験を提供することが求められている.\\

無線LAN環境では,多数のアクセスポイント(AP)を広範囲に配置し,利用者の移動や利用状況に応じて柔軟にネットワークを提供する必要がある.
従来は,ネットワーク管理者側がAPの設置位置最適化やチャネル割り当て,トラフィック制御といった手法により通信品質の改善を図ってきた.
しかし,建物構造による電波遮蔽や反射,管理外APによる干渉,利用者の予測困難な移動行動など,管理者の制御が及ばない要因も多く,
通信品質のばらつきや一時的な接続不良が依然として発生している.
このため,管理者主体のアプローチに加え,利用者側の行動を考慮した新たな通信品質改善手法が必要とされている.\\

本研究では,APと利用者端末との距離やユーザ数に加えて,APが使用するチャネルの使用率を評価指標として用いることで,より実環境に即した通信品質評価を行う.
この評価手法により,利用者に対して最小限の移動による適切なAP選択を促すユーザ行動支援手法を提案する.
提案手法の有効性を検証するため,ネットワークシミュレータ ns-3 を用いたシミュレーション評価を行った.
その結果,提案手法を適用した場合,利用者がランダムに移動した場合と比較して,システム全体のスループットが平均して向上することを確認した.
また,通信品質の改善と同時に,利用者の移動距離を抑制できることが示され,提案手法がエンタープライズ無線LAN環境において有効性のある通信品質改善手法であることを示した.
\end{abstract}

