%%%%%%%%%% 評価 %%%%%%%%%%
\section{評価}

\section{評価}
本章では,提案手法の有効性を検証するために実施したシミュレーション評価について述べる.
具体的には,ネットワークシミュレータを用いて無線 LAN 環境を再現し,提案手法に基づいてユーザが移動した場合と,ユーザがランダムに移動した場合とを比較した.
シミュレーション結果をもとに,システム全体のスループットおよびユーザの移動距離の観点から評価を行った.

\subsection{評価環境}
評価には,ネットワークシミュレータ ns-3(version 3.35)を用いた.\\
無線 LAN の規格としては,エンタープライズ環境で広く利用されているIEEE 802.11ax(2.4GHz 帯)を想定し,一般的なアクセスポイント(AP)をモデル化した.

シミュレーションは 1 回あたり 30 秒間とし,各条件について 100 回の試行を行った.
ユーザが最低限満たすべきスループットとして,Web 会議が可能な通信品質を想定し,最低許容スループットを 30 Mbps と設定した.

評価では,AP 数およびユーザ数が比較的少ない小規模環境と,AP 数およびユーザ数が多い大規模環境の 2 種類のシナリオを設定した.
各シナリオにおいて,新規ユーザはランダムな位置に出現し,既存ユーザはシミュレーション開始時点で AP に接続されているものとした.

\subsection{評価方法}
評価では,提案手法によって算出された移動先にユーザが移動した場合と,ユーザがランダムな方向に移動した場合とを比較対象とした.
いずれの場合も,ユーザは事前に設定された移動可能距離の範囲内でのみ移動するものとする.

評価指標としては,以下の 2 点を用いた.\\
1 つ目は,新規ユーザが移動する前後におけるシステム全体の合計スループットの改善率である.
これにより,提案手法が無線 LAN システム全体の通信品質向上に寄与しているかを評価する.

2 つ目は,ユーザごとの移動距離である.
提案手法では,ユーザに過度な移動を強いない「消極的移動」を方針としている.
そのため,スループットの改善だけでなく,移動距離がどの程度抑制されているかを併せて評価する.

これらの指標について,ランダム移動と提案手法による移動の結果を比較した.

\subsection{評価結果}
AP 数およびユーザ数が少ない小規模環境においては,
提案手法を適用した場合,ランダム移動と比較して
システム全体のスループット改善率の平均値が約 24.6\% 高い結果となった.
また,提案手法では移動距離が 0.5 m 以下に抑えられるユーザが一定数存在する一方で,
最大移動距離近くまで移動するユーザも確認された.

次に,AP 数およびユーザ数が多い大規模環境において評価を行った.
この環境では,ランダム移動と提案手法との間で,
システム全体のスループット改善率の平均値の差は約 2.5\% となった.
一方で,ユーザごとの移動距離に着目すると,
提案手法ではランダム移動と比較して移動距離が抑えられる傾向が確認された.
特に,提案手法における最小移動距離は,
ランダム移動における平均移動距離と同程度であり,
不要な移動を抑制できていることが示された.

\subsection{考察}
評価結果から,提案手法を適用することで,
ランダムに移動した場合と比較して,
システム全体のスループットを改善できることが確認できた.
特に小規模環境では,提案手法による効果が顕著に現れ,
ユーザ行動を適切に誘導することの有効性が示された.

一方で,大規模環境ではスループット改善率の差が小さくなる結果となった.
これは,ユーザ数の増加により無線資源の競合が激化し,
個々のユーザの移動による改善効果が相対的に小さくなったためであると考えられる.
また,初期位置によっては,最大移動距離まで移動しても要求スループットを満たせないユーザが存在することも確認された.

これらの結果から,提案手法は,ランダムな移動と比較してスループット改善と移動距離抑制の両立が可能である一方,
すべてのユーザに対して十分な通信品質を保証できるわけではないことが明らかとなった.
今後は,移動によっても改善が困難なユーザに対する追加的な対策を検討する必要がある.
