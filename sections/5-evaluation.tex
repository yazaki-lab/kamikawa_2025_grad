%%%%%%%%%% 評価 %%%%%%%%%%
\section{評価}
本章では,提案手法の有効性を検証するために実施したシミュレーション評価について述べる.
具体的には,ネットワークシミュレータを用いて無線 LAN 環境を再現し,提案手法に基づいてユーザが移動した場合と,ユーザがランダムに移動した場合とを比較した.

\subsection{実験環境}
評価には,ネットワークシミュレータ ns-3(version 3.35)を用いた.
無線 LAN の規格としては,エンタープライズ環境で広く利用されている IEEE 802.11ax(2.4GHz 帯)を想定し,一般的な AP をモデル化した.
シミュレーションは 1 回あたり 30 秒間とし,各条件について 100 回の試行を行った.
評価では,AP 数およびユーザ数が比較的少ない小規模環境と,AP 数およびユーザ数が多い大規模環境の 2 種類のシナリオを設定した.
各シナリオにおいて,新規ユーザはランダムな位置に出現し,既存ユーザはシミュレーション開始時点で AP に接続されているものとした.

\subsection{評価指標}
評価指標としては,以下の 2 点を用いた.
1 つ目は,新規ユーザが移動する前後におけるシステム全体の合計スループットの改善率である.
これにより,提案手法が無線 LAN システム全体の通信品質向上に寄与しているかを評価する.
2 つ目は,ユーザごとの移動距離である.
提案手法では,ユーザに過度な移動を強いない「消極的移動」を方針としている.
そのため,スループットの改善だけでなく,移動距離がどの程度抑制されているかを併せて評価する.

\subsection{実験 1:提案アルゴリズムの評価}
\subsubsection{目的}
本実験では,提案手法によって算出された移動先にユーザが移動した場合に,システム全体のスループット改善と移動距離抑制が両立できるかを検証する.
特に,環境規模の違いが提案手法の有効性に与える影響を明らかにすることを目的とする.
\subsubsection{方法}
評価では,提案手法によって算出された移動先にユーザが移動した場合と,ユーザがランダムな方向に移動した場合とを比較対象とした.
いずれの場合も,ユーザは事前に設定された移動可能距離の範囲内でのみ移動するものとする.
ユーザが最低限満たすべきスループットとして,Web 会議が可能な通信品質を想定し,最低許容スループットを 30 Mbps と設定した.
各条件について 100 回の試行を行い,システム全体のスループット改善率およびユーザごとの移動距離を測定した.

\subsubsection{結果}
図\ref{fig:throughput_small} に小規模環境におけるスループット改善率の比較を示す.
\begin{figure}[htbp]
  \centering
  \includegraphics[width=0.8\linewidth]{fig/throughput_boxplot.png}
  \caption{スループット改善率}
  \label{fig:throughput_small}
\end{figure}
提案手法を適用した場合,ランダム移動と比較してシステム全体のスループット改善率の平均値が約 $24.6\%$ 高い結果となった.\\

図\ref{fig:moving_small} に小規模環境におけるユーザごとの移動距離の分布を示す.
\begin{figure}[htbp]
  \centering
  \includegraphics[width=0.8\linewidth]{fig/moving_boxplot.png}
  \caption{移動距離の平均値}
  \label{fig:moving_small}
\end{figure}
提案手法ではユーザの移動が1m以下に抑えられていることが確認できる.
ランダム手法では移動距離が 0.5 m 以下に抑えられるユーザが一定数存在する一方で,最大移動距離近くまで移動するユーザも確認された.
これは,提案手法が各ユーザの状況に応じて適切な移動距離を算出していることを示している.\\

\subsubsection{考察}
評価結果から,提案手法を適用することで,ランダムに移動した場合と比較して,システム全体のスループットを改善できることが確認できた.
特に小規模環境では,提案手法による効果が顕著に現れ,スループット改善率が約 $24.6\%$ 向上した.
これは,ユーザ数が少ない環境では個々のユーザの移動が全体に与える影響が大きく,提案手法によるユーザ行動の適切な誘導が有効に機能したためであると考えられる.
一方で,大規模環境ではスループット改善率の差が約 $2.5\%$ と小さくなる結果となった.
これは,ユーザ数の増加により無線資源の競合が激化し,個々のユーザの移動による改善効果が相対的に小さくなったためであると考えられる.
また,初期位置によっては,最大移動距離まで移動しても要求スループットを満たせないユーザが存在することも確認された.
移動距離に関しては,小規模環境・大規模環境のいずれにおいても,提案手法はランダム移動と比較して不要な移動を抑制できていることが確認された.
これは,提案手法が「消極的移動」の方針に基づき,各ユーザに対して必要最小限の移動距離を算出していることを示している.
これらの結果から,提案手法は,ランダムな移動と比較してスループット改善と移動距離抑制の両立が可能である一方,すべてのユーザに対して十分な通信品質を保証できるわけではないことが明らかとなった.
今後は,移動によっても改善が困難なユーザに対する追加的な対策を検討する必要がある.

\subsection{実験 2:チャネル使用率推定の評価}

\subsubsection{目的}
本実験では,提案手法で用いるチャネル使用率推定モデルの精度を検証する.
具体的には,シミュレーション環境において様々なトラフィック負荷,通信距離,端末数の条件下でチャネル使用率を測定し,
提案する指数関数ベースの非線形回帰モデルが実測値を精度よく予測できるかを評価する.

\subsubsection{方法}
ネットワークシミュレータ ns-3(ver. 3.45)を用い,APと複数の端末からなる環境を構築した.
表\ref{tab:params}に主なシミュレーションパラメータを示す.

\begin{table}[htbp]
    \centering
    \caption{シミュレーションパラメータ}
    \label{tab:params}
    \begin{tabular}{ll}
        \toprule
        項目 & 設定値 \\
        \midrule
        プロトコル & IEEE 802.11n/ax  \\
        トランスポート層 & TCP \\
        トラフィック負荷 (Load) & 5 -- 200 Mbps \\
        端末数 (Stations) & 1 -- 40 \\
        通信距離 (Radius) & 5 -- 25 m \\
        \bottomrule
    \end{tabular}
\end{table}

シミュレーションでは,トラフィック負荷 $L$ [Mbps],平均 RSSI $R$ [dBm],端末数 $N$ をパラメータとして変化させ,
各条件におけるチャネル使用率 $U$ を測定した.
測定されたデータに対して,チャネル使用率がトラフィック負荷の増加に伴い飽和特性を示すことに着目し,
以下の指数関数型飽和モデルを用いた非線形回帰を行った.

\begin{equation}
    U(L, R, N) = Max \left( 1 - \exp\left( - (aL + bR + cN + d) \right) \right)
\end{equation}

ここで,$L$ はトラフィック負荷 [Mbps],$R$ は平均 RSSI [dBm],$N$ は端末数であり,
$Max$ はチャネル使用率の最大飽和値を表す.
理論的には,チャネル使用率は $0$~$100\%$ の範囲で推移し,最大値は $100\%$ に制約される.
しかし,実際の無線 LAN 環境やシミュレーションにおいてはチャネル使用率が $100\%$ に到達することは稀である.

実際に本シミュレーションデータに対して $Max=100$ と固定したモデルでフィッティングを行った場合,以下の式のようになり,
LやNの係数がRに比べ極端に大きく,各値の影響を正当に表しているとは言い難いと判断した.
\begin{equation}
    U(L, R, N) = 100 \left( 1 - e^{\left( - \left( 6.2070 L - 0.0057 R - 30.9139 N - 0.3386 \right) \right)} \right)
\end{equation}

このことから,チャネル使用率の実効的な上限値は環境条件に依存して変化すると考えられる.
そこで本研究では,最大飽和値 $Max$ も回帰係数の一つとして同時に推定することで,
高負荷領域における過大評価を抑制し,モデル全体の推定精度を向上させる.

本モデルは,トラフィック負荷の増加に伴いチャネル使用率が指数的に増加し,
環境依存の最大飽和値に漸近するというシミュレーション結果の傾向を表現するために導入する.

以下に,シミュレーション結果をグラフにプロットしたものを示す.
\begin{figure}[htbp]
  \centering
  \includegraphics[width=0.8\linewidth]{fig/3dgraph_measured.png}
  \caption{シミュレーションにおける測定結果}
  \label{fig:fitting_1}
\end{figure}

測定値に対して最小二乗法を用いて各係数($a$, $b$, $c$, $d$, $Max$)を推定し,
決定係数 $R^2$ によりモデルの適合度を評価した.

\subsubsection{結果}
最小二乗法によるフィッティングから得られた各係数および決定係数を表\ref{tab:result}に示す.

\begin{table}[htbp]
    \centering
    \caption{導出した係数}
    \label{tab:exp_saturation_result}
    \begin{tabular}{cr}
        \toprule
        係数 & 推定値 \\
        \midrule
        $a$ (Load) & $-0.0052$ \\
        $b$ (RSSI) & $-0.0062$ \\
        $c$ (Stations) & $0.2169$ \\
        $d$ (Bias) & $-0.4879$ \\
        $Max$ (Saturation) & $88.64~\%$ \\
        \midrule
        決定係数 $R^2$ & $0.9898$ \\
        \bottomrule
    \end{tabular}
\end{table}


これにより,チャネル使用率推定モデルは以下のように定式化される.
\begin{equation}
U(L, R, N) = 88.64 \left( 1 - \exp\left( - \left(
-0.0052 L
- 0.0062 R
+ 0.2169 N
- 0.4879
\right) \right) \right)
\end{equation}


表\ref{tab:result}より,トラフィック負荷 $L$ の係数 $a$ が正の値であることから,
トラフィック負荷の増加に伴いチャネル使用率が指数的に増加する傾向が確認できる.
一方,端末数 $N$ の係数 $c$ が負であることは,
端末数の増加によりチャネル競合が激化し,
単位時間あたりに観測される有効な送信割合が低下する影響を反映していると考えられる.

\subsubsection{考察}
実験結果から,提案する指数関数型飽和モデルは,
決定係数 $R^2 = 0.9802$ という高い精度でチャネル使用率を推定できることが確認された.
これは,本モデルがチャネル使用率の増加および飽和という非線形な特性を適切に捉えていることを示している.\\

また,高負荷領域においても推定値が過大評価されることなく,
最大飽和値 $Max$ に漸近する挙動が確認されており,
実際の無線 LAN 環境におけるチャネル使用率の物理的制約を反映したモデルとなっている.\\

ただし,本モデルは IEEE 802.11n/ax 環境に基づいて導出されたものであり,
異なる無線 LAN 規格やトラフィック特性を持つ環境に適用する場合には,
係数の再推定が必要となる可能性がある.
今後は,より多様な環境条件下でのモデルの汎化性能について検証を行う予定である.
