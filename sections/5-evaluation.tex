%%%%%%%%%% 評価 %%%%%%%%%%
\section{評価}
本章では,提案手法の有効性を検証するために実施したシミュレーション評価について述べる.
具体的には,ネットワークシミュレータを用いて無線 LAN 環境を再現し,提案手法に基づいてユーザが移動した場合と,ユーザがランダムに移動した場合とを比較した.

\subsection{実験環境}
評価には,ネットワークシミュレータ ns-3(version 3.35)を用いた.
無線 LAN の規格としては,エンタープライズ環境で広く利用されている IEEE 802.11ax(2.4GHz 帯)を想定し,一般的な AP をモデル化した.
シミュレーションは 1 回あたり 30 秒間とし,各条件について 100 回の試行を行った.
評価では,AP 数およびユーザ数が比較的少ない小規模環境と,AP 数およびユーザ数が多い大規模環境の 2 種類のシナリオを設定した.
各シナリオにおいて,新規ユーザはランダムな位置に出現し,既存ユーザはシミュレーション開始時点で AP に接続されているものとした.

\subsection{評価指標}
評価指標としては,以下の 2 点を用いた.
1 つ目は,新規ユーザが移動する前後におけるシステム全体の合計スループットの改善率である.
これにより,提案手法が無線 LAN システム全体の通信品質向上に寄与しているかを評価する.
2 つ目は,ユーザごとの移動距離である.
提案手法では,ユーザに過度な移動を強いない「消極的移動」を方針としている.
そのため,スループットの改善だけでなく,移動距離がどの程度抑制されているかを併せて評価する.

\subsection{実験 1:提案アルゴリズムの評価}
\subsubsection{目的}
本実験では,提案手法によって算出された移動先にユーザが移動した場合に,システム全体のスループット改善と移動距離抑制が両立できるかを検証する.
特に,環境規模の違いが提案手法の有効性に与える影響を明らかにすることを目的とする.
\subsubsection{方法}
評価では,提案手法によって算出された移動先にユーザが移動した場合と,ユーザがランダムな方向に移動した場合とを比較対象とした.
いずれの場合も,ユーザは事前に設定された移動可能距離の範囲内でのみ移動するものとする.
ユーザが最低限満たすべきスループットとして,Web 会議が可能な通信品質を想定し,最低許容スループットを 30 Mbps と設定した.
各条件について 100 回の試行を行い,システム全体のスループット改善率およびユーザごとの移動距離を測定した.

\subsubsection{結果}
図\ref{fig:throughput_small} に小規模環境におけるスループット改善率の比較を示す.
\begin{figure}
  \centering
  \includegraphics[htbp]{throughput_boxplot.png}
  \caption{Caption}
  \label{fig:throughput_small}
\end{figure}
提案手法を適用した場合,ランダム移動と比較してシステム全体のスループット改善率の平均値が約 $24.6\%$ 高い結果となった.\\

図\ref{fig:moving_small} に小規模環境におけるユーザごとの移動距離の分布を示す.
\begin{figure}
  \centering
  \includegraphics[htbp]{moving_boxplot.png}
  \caption{Caption}
  \label{fig:moving_small}
\end{figure}
提案手法ではユーザの移動が1m以下に抑えられていることが確認できる.
ランダム手法では移動距離が 0.5 m 以下に抑えられるユーザが一定数存在する一方で,最大移動距離近くまで移動するユーザも確認された.
これは,提案手法が各ユーザの状況に応じて適切な移動距離を算出していることを示している.\\

次に,図 Z に大規模環境におけるスループット改善率の比較を示す.
この環境では,ランダム移動と提案手法との間で,システム全体のスループット改善率の平均値の差は約 $2.5\%$ となった.
図 W に大規模環境におけるユーザごとの移動距離の分布を示す.
提案手法ではランダム移動と比較して移動距離が抑えられる傾向が確認された.
特に,提案手法における最小移動距離は,ランダム移動における平均移動距離と同程度であり,不要な移動を抑制できていることが示された.


\subsubsection{考察}
評価結果から,提案手法を適用することで,ランダムに移動した場合と比較して,システム全体のスループットを改善できることが確認できた.
特に小規模環境では,提案手法による効果が顕著に現れ,スループット改善率が約 $24.6\%$ 向上した.
これは,ユーザ数が少ない環境では個々のユーザの移動が全体に与える影響が大きく,提案手法によるユーザ行動の適切な誘導が有効に機能したためであると考えられる.
一方で,大規模環境ではスループット改善率の差が約 $2.5\%$ と小さくなる結果となった.
これは,ユーザ数の増加により無線資源の競合が激化し,個々のユーザの移動による改善効果が相対的に小さくなったためであると考えられる.
また,初期位置によっては,最大移動距離まで移動しても要求スループットを満たせないユーザが存在することも確認された.
移動距離に関しては,小規模環境・大規模環境のいずれにおいても,提案手法はランダム移動と比較して不要な移動を抑制できていることが確認された.
これは,提案手法が「消極的移動」の方針に基づき,各ユーザに対して必要最小限の移動距離を算出していることを示している.
これらの結果から,提案手法は,ランダムな移動と比較してスループット改善と移動距離抑制の両立が可能である一方,すべてのユーザに対して十分な通信品質を保証できるわけではないことが明らかとなった.
今後は,移動によっても改善が困難なユーザに対する追加的な対策を検討する必要がある.

\subsection{実験 2:チャネル使用率推定の評価}

\subsubsection{目的}
本実験では,提案手法で用いるチャネル使用率推定モデルの精度を検証する.
具体的には,シミュレーション環境において様々なトラフィック負荷,通信距離,端末数の条件下でチャネル使用率を測定し,
提案するシグモイド関数ベースの非線形回帰モデルが実測値を精度よく予測できるかを評価する.

\subsubsection{方法}
ネットワークシミュレータ ns-3(ver. 3.45)を用い,単一のアクセスポイント(AP)と複数のステーション(STA)からなる基本サービスセット(BSS)を構築した.
表\ref{tab:params}に主なシミュレーションパラメータを示す.

\begin{table}[htbp]
    \centering
    \caption{シミュレーションパラメータ}
    \label{tab:params}
    \begin{tabular}{ll}
        \toprule
        項目 & 設定値 \\
        \midrule
        プロトコル & IEEE 802.11n/ax  \\
        トランスポート層 & TCP \\
        トラフィック負荷 (Load) & 5 -- 200 Mbps \\
        端末数 (Stations) & 1 -- 40 \\
        通信距離 (Radius) & 5 -- 25 m \\
        \bottomrule
    \end{tabular}
\end{table}

シミュレーションでは,トラフィック負荷 $L$ [Mbps],平均 RSSI $R$ [dBm],端末数 $N$ をパラメータとして変化させ,
各条件におけるチャネル使用率 $U$ を測定した.
測定されたデータに対して,以下のシグモイド関数を用いた非線形回帰モデルを適用した.

\begin{equation}
    U(L, R, N) = \frac{Max}{1 + \exp(-(aL + bR + cN + d))}
\end{equation}

ここで,$L$はトラフィック負荷 [Mbps],$R$は平均 RSSI [dBm],$N$は端末数である.
このモデルは,チャネル使用率がトラフィック負荷の増加に対して飽和特性を示すという観測結果に基づいている.

以下に,シミュレーション結果をグラフにプロットしたものを示す.
\begin{figure}
  \centering
  \includegraphics[htbp]{cu_0119_1st.png}
  \caption{Caption}
  \label{fig:fitting_1}
\end{figure}

この値を基に,最小二乗法を用いて各係数($a$, $b$, $c$, $d$, $Max$)を推定し,
決定係数 $R^2$ によってモデルの適合度を評価した.

\subsubsection{結果}
最小二乗法を用いたフィッティングにより得られた各係数および決定係数を表\ref{tab:result}に示す.

\begin{table}[htbp]
    \centering
    \caption{導出されたモデルの係数}
    \label{tab:result}
    \begin{tabular}{cr}
        \toprule
        係数 & 推定値 \\
        \midrule
        $a$ (Load) & 8.0695 \\
        $b$ (RSSI) & -0.0244 \\
        $c$ (Stations) & -39.8982 \\
        $d$ (Intercept) & -3.5333 \\
        $Max$ (Saturation) & 86.42\% \\
        \midrule
        決定係数 $R^2$ & 0.9802 \\
        \bottomrule
    \end{tabular}
\end{table}

これにより,チャネル使用率推定モデルは以下のように定式化される.
\begin{equation}
    U(L, R, N) = \frac{86.42}{1 + \exp\left(-\left(8.0695 L - 0.0244 R - 39.8982 N - 3.5333\right)\right)}
\end{equation}


表\ref{tab:result}の係数から,トラフィック負荷 $L$ の係数 $a$ が正の値(8.0695)であることがわかる.
これは,トラフィック負荷の増加に伴ってチャネル使用率が上昇することを示している.
一方,端末数 $N$ の係数 $c$ が負の値(-39.8982)であることは,
端末数が増加すると個々の端末あたりの送信機会が減少し,見かけ上のチャネル使用率が低下する傾向を反映していると考えられる.

\subsubsection{考察}
実験結果から,提案するシグモイド関数ベースの非線形回帰モデルは,
決定係数 $R^2 = 0.9802$ という高い精度でチャネル使用率を推定できることが確認された.
これは,モデルが実測値の約 98\% の変動を説明できることを意味しており,
提案手法における移動先算出の基礎となるチャネル使用率推定が十分に信頼できることを示している.\\

決定係数から,提案モデルは様々なトラフィック負荷,通信距離,端末数の条件下で適用可能であると言える.
特に,トラフィック負荷が高い領域においても飽和特性を適切に捉えられている.\\

ただし,本モデルは IEEE 802.11n/ax 環境での測定結果に基づいて構築されているため,
他の無線 LAN 規格や異なる環境条件下では係数の再調整が必要となる可能性がある.
また,1端末が極端に大きな通信をする場合には,想定と異なる挙動をする可能性がある.\\
よって,今後は,より多様な環境条件下でのモデルの汎化性能を検証する必要がある.