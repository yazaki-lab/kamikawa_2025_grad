%%%%%%%%%% 実装に関して %%%%%%%%%%
\section{提案手法}

\subsection{提案手法の概要}
本研究では,エンタープライズ無線LAN環境における通信品質の改善を目的として,ネットワーク管理側の制御のみに依存するのではなく,利用者自身の能動的な行動変容を促すことにより,通信環境全体の最適化を図るユーザ行動支援手法を提案する.
近年,オフィスビルや大学キャンパスなどのエンタープライズ環境では,無線LANが業務や学習を支える基盤インフラとして広く利用されている一方で,利用者数や端末数の増加に伴い,通信品質のばらつきやスループット低下が問題となっている\cite{buffalo727WiFi}.

従来の無線LAN管理における最適化アプローチは,主にアクセスポイント(AP)の設置位置の最適化,チャネル割り当ての調整,送信出力制御といった,管理者主導の技術的施策に焦点が当てられてきた.
これらの手法は,静的あるいは準静的な環境を前提とした場合には有効であるものの,実際の運用環境では必ずしも十分な効果を発揮しない場合が多い.

その要因として,建物構造による電波の遮蔽や反射,不特定多数のユーザによる予測困難な移動,さらには来訪者が持ち込むモバイルルータやテザリング端末といった管理外の通信端末による干渉など,時間的・空間的に変動する要素が挙げられる.
これらの動的な要因を,管理者側の制御のみでリアルタイムに把握し,完全に解決することは困難である.

そこで本研究では,通信システムを構成する重要な要素として「ユーザ」に着目し,ネットワーク側が把握可能な環境情報と,ユーザ自身が提供する情報を組み合わせることで,ユーザに対して適切な行動指針を提示する新たなアプローチを検討する.
具体的には,ネットワーク管理側から取得可能な情報に基づき,ユーザに対して最適な接続先APや,通信品質の向上が期待できる移動先を提示することで,高品質な通信体験の実現を目指す.

提案アルゴリズムの特徴を以下にまとめる.
\begin{itemize}
  \item APとユーザ間の距離に加え,チャネル使用率および接続ユーザ数を考慮したAP選択を行う点
  \item ユーザの許容移動距離および最低要求スループットを制約条件として明示的に導入している点
  \item 新規ユーザ単体ではなく,システム全体のスループット最大化を目的としている点
\end{itemize}

これにより,実環境における電波干渉や負荷の偏りを考慮した,現実的かつ合理的なAP選択および移動誘導が可能となる.

\subsection{提案アルゴリズム}

本研究では,エンタープライズ無線LAN環境において新たに接続を行う利用者に対し,通信品質の向上を目的として,最適なアクセスポイント(AP)および移動方向を提示するアルゴリズムを提案する.
提案手法は,APと利用者端末の距離に基づいて接続先を決定する従来手法を拡張し,APが使用しているチャネルの使用率や接続ユーザ数といった無線環境の混雑状況を考慮する点に特徴がある.
本研究では,通信品質(QoE)の評価指標としてスループットを採用し,新規ユーザの接続によってシステム全体のスループットが最大化されることを目的とする.

\subsubsection{入力情報と前提条件}

提案アルゴリズムは,新規ユーザ$u$に対して以下の情報が与えられていることを前提とする.また,無線LAN環境内に存在するAPの集合を$A$とし,各AP $a \in A$に対して以下の情報が既知であると仮定する.
表\ref{tab:input_info}に入力情報の一覧を示す.なお,スループットの平均値は接続ユーザのスループットの調和平均として計算されるものとする.

\begin{table}[h]
\centering
\caption{提案アルゴリズムの入力情報}
\label{tab:input_info}
\begin{tabular}{lll}
\hline
カテゴリ & パラメータ & 説明 \\ \hline \hline
\multirow{3}{*}{ユーザ情報}
  & $P_u$ & 現在位置 \\
  & $d_{th}$ & 許容最大移動距離 \\
  & $\theta_{th}$ & 最低要求スループット \\ \hline
\multirow{4}{*}{AP情報}
  & $P_a$ & APの位置座標 \\
  & $n_a$ & 接続中のユーザ数 \\
  & $U_c$ & 使用チャネルの使用率 \\
  & $\theta^{before}_a$ & 接続前の平均スループット \\ \hline
\end{tabular}
\end{table}

なお,本研究では位置情報の取得方法や入力インタフェースについては議論せず,ユーザが自己申告により現在位置を入力できるものと仮定する.

\subsubsection{接続候補APの抽出}

新規ユーザが移動可能な範囲内に存在するAPのみを接続候補として抽出する.
ユーザの移動は移動ベクトル$m$により表され,移動後のユーザ位置は$P_u - m$とする.
このとき,移動後のユーザとAP $a$との距離$d_{u,a}$は以下の式で定義される.

\begin{equation}
d_{u,a} = \left| P_a - (P_u - m) \right|
\end{equation}

距離$d_{u,a}$が許容最大移動距離$d_{th}$以下となるAPの集合を,接続候補AP集合$A_{cand}$とする.

\subsubsection{接続前後のスループット推定}

次に,各接続候補APに対して,新規ユーザ接続前後のスループットを推定する.
AP $a$に接続されている既存ユーザの平均スループット$\theta^{before}_a$は,接続ユーザのスループットの調和平均として与えられているものとする.

新規ユーザがAP $a$に接続した場合のスループット$\theta^{after}_{a,m}$は,APとユーザ間の距離およびチャネル使用率$U_c$を考慮して推定される.
このとき,新規ユーザのスループットが最低要求スループット$\theta_{th}$を下回る場合,または移動距離が許容最大移動距離$d_{th}$を超える場合には,当該APを接続候補から除外する.

\subsubsection{評価スコアの算出}

残った接続候補APに対して,以下の4つの評価指標を用いて総合スコアを算出する.
\begin{enumerate}
  \item 接続後スループットに基づくスループットスコア
  \item ユーザとAPの距離に基づく距離スコア
  \item APが使用するチャネルの使用率に基づくチャネル使用率スコア
  \item APに接続されているユーザ数に基づく接続ユーザ数スコア
\end{enumerate}

各スコアは正規化された値として算出され,重み係数$w_i$を用いて以下の式により総合スコア$Score(a)$を計算する.

\begin{equation}
Score(a) = \sum_i w_i \cdot score_i(a)
\end{equation}

\subsubsection{最適APおよび移動ベクトルの決定}

提案アルゴリズムでは,総合スコアが最大となるAPを単純に選択するのではなく,総合スコア上位の複数APを候補として扱う.
その上で,各候補APに対して,新規ユーザ接続後のシステム全体のスループットを算出し,これが最大となるAP $a^*$と移動ベクトル$m^*$を最終的な出力とする.

システム全体のスループット$\Theta^{after}_{a,m}$は,以下の式で表される.

\begin{equation}
\Theta^{after}_{a,m}
= \theta^{after}_{a,m} + \sum_{i \neq a} \theta^{before}_i
\end{equation}

以上で述べた提案アルゴリズムを擬似コードとしてAlgorithm\ref{alg:optimalAP}にまとめる.
%
\begin{algorithm}[tbh]
    \begin{algorithmic}[1]
        \State \textbf{Input:}
        $P_u$(ユーザ現在位置),
        $d_{th}$(最大許容移動距離),
        $\theta_{th}$(最低要求スループット),
        $A$(AP集合)

        \State \textbf{Output:}
        $a^*$(選択AP),
        $m^*$(推奨移動ベクトル)

        \Statex
        \Statex \Comment{--- AP候補抽出 ---}
        \State $A_{cand} \gets \{ a \in A \mid \|P_a - P_u\| \le d_{th} \}$

        \Statex
        \Statex \Comment{--- 接続可否判定 ---}
        \For{each $a \in A_{cand}$}
            \State Compute $\theta_a^{before}$ using harmonic mean of existing users
            \State Estimate $b_{new,a}$ from RSSI at $P_u$
            \State Compute $\theta_{a,0}^{after}$ assuming no movement
            \State Compute $\theta_{new,a,0}$ for the new user
            \If{$\theta_{new,a,0} < \theta_{th}$}
                \State $A_{cand} \gets A_{cand} \setminus \{a\}$
            \EndIf
        \EndFor

        \Statex
        \Statex \Comment{--- スコア計算 ---}
        \For{each $a \in A_{cand}$}
            \State Compute $score_{\theta_a}$ from $\theta_{a,0}^{after}$
            \State Compute $score_{d_{u,a}}$ from $\|P_a - P_u\|$
            \State Compute $score_{U_c}$ from channel utilization
            \State Compute $score_{n_a}$ from number of users
            \State $Score(a) \gets \sum_i w_i \cdot score_i(a)$
        \EndFor

        \Statex
        \Statex \Comment{--- 最終選択 ---}
        \State $A_{cand}' \gets$ top-3 APs by $Score(a)$
        \State $\{a^*, m^*\} \gets
        \arg\max_{a \in A_{cand}',\, \|m\|\le d_{th}}
        \theta_{a,m}^{after}$

        \State \Return $\{a^*, m^*\}$
    \end{algorithmic}
    \caption{提案アルゴリズム}
    \label{alg:optimalAP}
\end{algorithm}


\subsection{チャネル使用率の推定}

提案アルゴリズムでは,AP選択の評価指標の一つとしてチャネル使用率を用いる.
2.5節で述べた通り,無線LAN環境において,チャネル使用率はAPがどの程度通信に使用されているかを表す重要な指標であり,通信品質やスループットの低下を把握する上で不可欠である.
802.11k 規格では,チャネル使用率以外にもいくつかの無線資源管理に関する情報が定義されており,無線LAN クライアントは,これらの情報を元に最適なAPを選択することができる.\\
しかしながら,一部の無線LAN 環境においては,AP の機能・性能やネットワーク構成の制約などの理由から,このチャネル使用率に関する情報が提供されない場合や,取得が困難な場合がある.
またエンドユーザーの端末といった改修が憚られる端末においては,特別なドライバなどが導入できない場合も多い.\\

以上の課題から,提案手法を満足に実装するためには,チャネル使用率を推定する手法が求められる.本説では,このチャネル使用率の推定手法について述べる.
無線LAN環境において,チャネル使用率に影響を与える要因は様々ある.
トラフィック負荷,無線LANクライアントの数,APとクライアントの距離などはその主因となり得るが,これ以外にも多くの要因が想定できる.
また,外乱による電波環境の変化などもチャネル使用率に影響を与える.
これらすべての要因を抽出し,解析的手法でチャネル使用率推定手法をモデル化することは困難である.
そこで本研究では,ネットワークシミュレータを用いて様々なパターンの無線環境をシミュレートし,その結果から,チャネル使用率に影響を与える主要因を分析する.

本研究では,ネットワークシミュレータ ns-3(Ver.~3.45)\cite{ns3website} を用いて無線LAN環境を模擬した.
さらに,チャネル使用率に影響を与える主要因と考えられるトラフィック負荷,無線LANクライアント数,APとクライアントの距離などの要因を変化させた場合のチャネル使用率を測定することで,チャネル使用率を推定するうえで重要なパラメータを特定する数値実験を行った.

実験に使用するns-3によるシミュレーションの基本構成を図\ref{fig:apsetup}に示す.
%
\begin{figure}[tb]
    \centering
    \includegraphics[width=0.45\linewidth]{./fig/apsetup.png}
    \caption{シミュレーションの基本構成.図は4台の無線LANクライアントが接続されている場合を示す.}
    \label{fig:apsetup}
\end{figure}
%
また,シミュレーションの基本条件は次のとおりである.
%
\begin{itemize}
    \item 無線LAN規格 IEEE 802.11ax を用いる
    \item AP は 1 台とし,この AP に複数の無線LANクライアントが接続されている
    \item AP と各無線 LAN クライアントの距離はすべて同じとする
    \item トランスポート層には TCP を用いる
    \item パケットサイズを 1,400 バイトとする
    \item トラフィック負荷となる通信として,APからすべての無線LANクライアントに対して同時に10Mbpsの負荷が発生する.これはフルHD動画のストリーミングに相当する負荷である
\end{itemize}

本節では,チャネル使用率を推定する関数を$U(N, L, R)$と表記する.
ここで,$N$はAPに接続している無線LANクライアント数 (台),$L$はAPから各無線LANクライアントに対して発生させるトラフィック負荷 (Mbps),$R$はAPと各無線LANクライアント間の平均受信信号強度指標 RSSI (dBm) をそれぞれ表す.
本節では,これらの3つの要因がチャネル使用率に大きな影響を与えると仮定し,チャネル使用率推定モデルの構築を試みた.
これらの指標は,実際の無線LAN環境においても,APおよび無線LANクライアントの機能により,比較的容易に収集可能である.

シミュレーションのパラメーターとしては表\ref{tab:params}を用いる.
%
\begin{table}[tb]
    \centering
    \caption{シミュレーションパラメーター}
    \label{tab:params}
    \begin{tabular}{ll}
        \toprule
        パラメーター & 値 \\
        \midrule
        トラフィック負荷 $L$ (Mbps) & 10 \\
        無線LANクライアント数 $N$ (台) & 1, 5, 10, 15, 20 \\
        通信距離 $D$ (m) & 0, 5, 10, 15, 20\\
        \bottomrule
    \end{tabular}
\end{table}
%
ここで,通信距離$D$は主にRSSIを変化させることを意図したパラメーターである.
シミュレーションにおいて通信距離$D$を変化させることで,RSSIの変化を観測する.
RSSIは信号強度であり,APと無線LANクライアント間の距離が増加するにつれて悪化する傾向がある.
RSSIは値が大きいほど電波受信状況が良好であることを示す.


\subsubsection{シミュレーション結果}

図\ref{fig:ChUtil_vs_RSSI_L10M}に,前述の基本構成および基本条件のシミュレーションにおいて,無線LANのクライアント数を変化させた場合の,平均RSSIとチャネル使用率の関係を示す.
%
\begin{figure*}[tb]
    \centering
    \subfigure[無線LANクライアント数 1]{
        \includegraphics[width=0.45\linewidth]{./fig/ChUtil_vs_RSSI-S1_L10M.png}
        \label{fig:ChUtil_vs_RSSI-S1_L10M}
    }
    \subfigure[無線LANクライアント数 5]{
        \includegraphics[width=0.45\linewidth]{./fig/ChUtil_vs_RSSI-S5_L10M.png}
        \label{fig:ChUtil_vs_RSSI-S5_L10M}
    }

    \subfigure[無線LANクライアント数 10]{
        \includegraphics[width=0.45\linewidth]{./fig/ChUtil_vs_RSSI-S10_L10M.png}
        \label{fig:ChUtil_vs_RSSI-S10_L10M}
    }
    \subfigure[無線LANクライアント数 15]{
        \includegraphics[width=0.45\linewidth]{./fig/ChUtil_vs_RSSI-S15_L10M.png}
        \label{fig:ChUtil_vs_RSSI-S15_L10M}
    }
    \subfigure[無線LANクライアント数 20]{
        \includegraphics[width=0.45\linewidth]{./fig/ChUtil_vs_RSSI-S20_L10M.png}
        \label{fig:ChUtil_vs_RSSI-S20_L10M}
    }
    \caption{APからすべての端末にトラフィック負荷10Mbpsを発生させた場合における平均RSSIとチャネル使用率の変化の関係.RSSIはAPと端末間の距離に依存する.}
    \label{fig:ChUtil_vs_RSSI_L10M}
\end{figure*}
%
図\ref{fig:ChUtil_vs_RSSI_L10M}は,平均RSSIとチャネル使用率の関係を示しており,横軸に平均RSSI(dBm),縦軸にチャネル使用率(\%)を取っている.
一般的に,複数の無線LAN機器メーカーの設定ガイドでは,RSSIが-70 dBm以上の場合を良好,-70〜-80 dBmを通常,-80 dBm以下を不良とする指標が示されている\cite{apple_deployment,juniper_rssi_roaming}.
また,環境差はあるものの,ノイズフロアは概ね-90 dBm程度と見積もられることが多い.
このため,RSSIが-70 dBm以上であれば,信号対雑音比(SNR)は20 dB以上となり,無線通信において十分な受信品質が確保されている状態であると考えられる.

通信理論において,通信速度はシャノンの定理に基づき,SNRの増加に伴って向上することが知られている.
無線LANでは,SNRに応じてModulation and Coding Scheme(MCS)が選択され,リンク速度が動的に変化する\cite{snr-to-mcs}.
MCSは番号が大きくなるほど高次の変調方式および高効率な符号化方式を用いるため,SNRが高い条件下ではデータレートの向上が期待できる.
このような特性を踏まえることで,図\ref{fig:ChUtil_vs_RSSI_L10M}(a),(b)に示すRSSIとチャネル使用率の関係を説明できる.

図\ref{fig:ChUtil_vs_RSSI_L10M}より,無線LANクライアント数が増加するにつれて,チャネル使用率が高くなる傾向が確認できる.
特に,クライアント数が10台,15台,20台の場合では,RSSIの値に依らず,チャネル使用率はおおむね80%以上で推移している.
これは,多数のクライアントが同一の無線チャネルを共有することで,チャネルの帯域がほぼ飽和状態に達しているためと考えられる.

Cisco Systems社の資料\cite{cisco-comm-guide-mcsindex}によれば,実効スループットは以下の経験則で見積もることができる.
%
\begin{equation}
\text{予想スループット} = \text{データレート} \times 0.7
\end{equation}
%
ここで,図\ref{fig:ChUtil_vs_RSSI_L10M}においてRSSIが-30 dBmの場合を考えると,SNRとMCSの対応関係\cite{snr-to-mcs}から,802.11acを参考にするとMCS 9以上に相当すると推定される.
この条件における802.11axの1空間ストリーム,チャネル幅20 MHzのデータレートは約109 Mbpsである.
このデータレートを複数の端末で共有することを考慮すると,高いチャネル使用率が維持されることは妥当である.

以上より,RSSIが-60 dBm以上であれば受信電波強度自体は十分に良好であるものの,受信品質の向上が必ずしもチャネル使用率の低下には結びついていないことが分かる.
したがって,本シミュレーション条件下では,クライアント数が10台以上の場合,端末をAPに近づけてRSSIを改善しても,チャネル使用率の改善効果は限定的であると考えられる.

一方で,無線LANクライアント数が1台および5台の場合には,RSSIの増加に伴いチャネル使用率が低下する傾向が見られる.
クライアント数が5台の場合は,RSSIの改善に応じてチャネル使用率が緩やかに減少しており,端末位置の調整による一定の改善効果が期待できる.
さらに,クライアント数が1台の場合には,RSSIの増加に対してチャネル使用率が急激に低下しており,APに近づけることで大幅なチャネル使用率の改善が可能であると考えられる.
% 図\ref{fig:ChUtil_vs_RSSI_L10M}中のグラフにおいて,横軸は平均RSSI (dBm),縦軸はチャネル使用率 (\%) を示す.
% 複数の無線LAN機器メーカーの設定ガイドを参照すると,RSSIが-70 dBm以上であれば良好,-70〜-80 dBmであれば通常,-80 dBm以下であれば不良とするなどの目安が示されている\cite{apple_deployment,juniper_rssi_roaming}.
% 個々の環境に依るものの,一般にノイズフロアは-90 dBm程度と見積もられる.
% RSSIが-70 dBm以上あれば,信号対雑音比 (SNR) は20 dB以上となり,一般的に良好な電波強度であると言える.

% 一般に通信速度はシャノンの定理に従って,SNRが高くなるほど向上する.
% 無線LAN環境では,SNR毎にModulation and Coding Scheme (MCS) が決定され,リンク速度が動的に変化する\cite{snr-to-mcs}.
% MCSは続く数字が大きいほど高速通信が可能な変調方式を採用していることから,SNRが高いほどデータレートは向上する.
% 図\ref{fig:ChUtil_vs_RSSI_L10M}(a),(b)に示すRSSIとチャネル使用率の結果を説明できる.

% 図\ref{fig:ChUtil_vs_RSSI_L10M}より,無線LANクライアント数が増加するほどチャネル使用率が高くなる傾向であることがわかる.
% 無線LANクライアント数が10, 15, 20の場合においては,RSSIの数値に関わらずチャネル使用率は約80\%以上となっている.
% これは,無線LANクライアント数が多い場合,伝送空間である無線チャネルそのもの帯域をすべて使用してしまっているためと考えられる.
% Cisco Systems社\cite{cisco-comm-guide-mcsindex}によると,経験則的な予想スループットは次の通り求められる.
% %
% \begin{equation}
%     \text{予想スループット} = \text{データレート} \times 0.7
% \end{equation}
% %
% ここで,図\ref{fig:ChUtil_vs_RSSI_L10M}におけるRSSIが-30 dBmの時のデータレートをおおよそ見積もると,SNRとMCSの対応表\cite{snr-to-mcs}より,802.11acの場合を流用すると,MCS 9以上とわかる.
% このときの802.11axにおける1空間ストリーム・チャネル幅20 MHzのデータレートは約109 Mbpsである.
% 複数台の端末でこのデータレートを共有することを考慮すると,この条件下ではチャネル利用率が高止まりすることが理解できる.
% %
% 上述のRSSIの目安に基づくと,RSSIが-60 dBm以上であれば電波受信状況自体は良好であると判断できるが,電波受信状況の改善がチャネル使用率の改善に寄与していないことが読み取れる.
% このことから,今回のシミュレーション条件においては,クライアント数が10台以上になると,無線LANクライアントの位置をAPに近づけることによって電波受信状況を改善しても,チャネル使用率の改善は期待できないことがわかる.

% 一方,無線LANクライアント数が1台および5台の場合においては,RSSIの数値が大きいほどチャネル使用率が低くなる傾向が見られる.
% 無線LANクライアント数が5の場合は,RSSIの改善に対してチャネル使用率も緩やかに低下している.
% この場合は,クライアントをAPに近づけることで,チャネル使用率の改善が期待できる.
% 無線LANクライアント数が1の場合は,RSSIの改善に対してチャネル使用率が指数関数的に低下している.
% この場合は,クライアントをAPに近づけることで,チャネル使用率の大幅な改善が期待できる.