%%%%%%%%% 終わりに %%%%%%%%%%
\section{おわりに}

\subsection{まとめ}
本研究では,エンタープライズ無線LAN環境における通信品質の改善を目的として,
利用者の行動に着目したユーザ行動支援手法を提案した.
従来の無線LAN環境における通信品質改善は,アクセスポイントの設置や設定調整など,ネットワーク管理者側による技術的対策が中心であった.
しかし,実際の利用環境では,利用者の移動や利用行動が通信品質に大きな影響を与えており,管理者側の制御のみでは十分な改善が得られない場合が多い.

そこで本研究では,アクセスポイントと利用者端末との距離に加えて,アクセスポイントが使用するチャネルの使用状況を考慮し,
利用者に対して最小限の行動で通信品質が改善されるような行動支援手法を検討した.
提案手法の有効性を検証するため,ネットワークシミュレータ ns-3 を用いたシミュレーション評価を行い,
利用者がランダムに移動する場合と比較した.

その結果,提案手法を適用することで,ネットワーク全体のスループットが向上し,通信品質のばらつきが抑えられることを確認した.
また,利用者の移動を最小限に抑えた状況においても一定の改善効果が得られることから,
利用者の負担を増やさずに通信品質を改善できる可能性が示された.
これらの結果より,利用者の行動を考慮した通信品質改善手法が,エンタープライズ無線LAN環境において一定の効果を持つことが示唆された.


\subsection{今後の課題}
本研究には,いくつかの課題が残されている.\\
第一に,本研究で用いた評価はシミュレーション環境に限定されており,
実際の無線LAN環境における検証は行っていない点である.
実環境では,建物構造による電波遮蔽や反射,外部ネットワークからの干渉,端末性能の違いなど,シミュレーションでは十分に再現できない要素が存在する.
今後は,実環境での実験や測定を通じて,提案手法の実用性を検証する必要がある.\\

第二に,本研究では利用者の行動を比較的単純なモデルで扱っている点が挙げられる.
実際の利用者は,通信品質だけでなく,作業内容や滞在時間,周囲の混雑状況など,さまざまな要因を考慮して行動する.
今後は,より現実的な利用者行動モデルを導入し,
多様な利用シナリオにおける評価を行うことが求められる.\\

第三に,利用者への行動提示方法についても検討の余地がある.
本研究では,行動支援の有効性を主に通信品質の観点から評価したが,
実際に利用者に提示する際には,分かりやすさや受け入れやすさも重要な要素となる.
通知方法や表示内容,行動を促すための仕組みについて検討することで,より実用的なシステムの実現が期待できる.

今後は,これらの課題に取り組むことで,
利用者の行動を考慮した通信品質改善手法を
より現実的かつ実用的なものへと発展させていきたい.
