\section{関連研究}
無線LAN環境の改善に向けた取り組みは数多く報告されている.
Ranasingheら\cite{Ranasinghe2021}やWu\cite{Wu2024}らは,各APに接続される端末数のロードバランシングに関する手法を提案している.
ロードバランシングは無線LAN環境を改善するために管理者側で行う対策としては基本的なものである.
各APに接続される端末の台数をできるだけ均一にすることで,APごとの通信負荷を分散し,特定のAPに過度な負荷が集中することを防ぐ.
一方でユーザの位置的分布や行動特性を変化させるものではないため,実環境における通信品質の改善効果には限界がある.
%主としてAP配置に焦点を当てており,ユーザの行動特性や端末側の視点を考慮したアプローチについては十分に検討されていない.
既製品としても,ユーザの通信に関するQoEを評価・可視化する機能を持つものはあるが,具体的に通信環境を改善する機能としては,AP配置やチャネル割り当ての最適化にとどまっている.

Rowdenらの研究では,電波強度をVR技術を用いて可視化し,利用者が最適なAPを選択可能とする手法が提案されている\cite{Rowden2023}.
この手法はVRゴーグル等の専用機器を必要とするため,実環境で利用者に提供する機能としては現実的ではない.
現実的な運用環境においては,利用者が自身の端末のみで通信環境を改善するための行動を確認できることが重要な要件となる.

本研究が目指すように,利用者に行動を促すことで,無線LAN環境を改善する取り組みもある\cite{Miyata2012}.
Miyataらは,利用者の協調的な移動を考慮した新たなAP選択手法を提案している.
同研究は,各利用者がそれぞれ指定する「移動可能距離」と「許容スループット」を条件とし,無線LANシステム全体のスループットを最大化するための移動を利用者に促すものである.
この手法は,スループットを評価指標として,APと利用者の距離のみに着目した手法であり,電波干渉や物理的障害物,他利用者数の動的変化といった要因は考慮されていない.
また,シミュレーションのみで手法を評価しており,実環境における適用についてはその効果は未解明である.
この手法はAPや利用者の位置を特定する方法としてAPとは異なるセンサーを多数設置することを想定しており,実環境においてはコスト面での制約が大きい点も課題である.

一方で,無線LAN環境の状態を適切に把握し,ユーザへの行動支援に活用するためには,通信品質を定量的に評価する指標が必要となる.
この点において,チャネル利用率は無線LANの混雑状況や通信品質を直接的に反映する有用な指標である.
チャネル利用率に関連する研究にはBianchiらによるモデルがある\cite{Bianchi2000}.
この研究では理想的な環境を定義した上で,無線LANにかかる処理を定式化することで,チャネル利用率に関連するようなパラメータを解析的に求めている.
%
より直接的なチャネル利用率の推定としては,Zhaoらによるキャリアセンスに着目した手法がある\cite{Zhao2013}.
Zhaoらはチャネルの利用されていないIDLE時間を新たなセンシング手法で推定し,実効的な最大スループットを求める手法を提案した.
この手法は他手法に比べて,より実験値に近い値を算出できていることが示されている.
%
端末の情報のみを用いた無線環境の推定としては,Rossiらによる実験がある\cite{Rossi2014}.
Rossiらは特殊機材などを用いず,Linux 向けカスタムドライバを導入することでのみ得られる情報を用いて,無線LAN環境の推定を行っている.
%
Remote Access Network(RAN)分野でも伝送路の状態推定が行われており,Guttermanら\cite{Gutterman2019}による研究がある.
RANスライシングに代表される5Gネットワークでは,適切なサービス提供を行うため,伝送路の状態とユーザの需要を適切に識別・推定したうえでスケジューリングを行う必要がある.
この課題に対して,Guttermanらは古典的な統計時系列モデルと機械学習であるLSTMを組み合わせた手法を提案している.